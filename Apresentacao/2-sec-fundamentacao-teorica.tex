\section{Fundamentação Teórica}\label{sec:fundamentacao-teorica}

% ----------------- SLIDE 14 --------------------------------
\begin{frame}{Cliente/Servidor}
	\begin{block}{Cliente}
		``Um solicitante de informações em rede, normalmente um computador ou 
		estação de trabalho, que pode consultar o banco de dados e/ou outras 
		informações de um servidor.'' (STALLINGS, 2005)
	\end{block}
	\begin{block}{Servidor}
		``Um computador, normalmente uma estação de trabalho poderosa ou um 
		\textit{mainframe} que abriga informações para manipulação por clientes 
		em rede.'' (STALLINGS, 2005)
	\end{block}
\end{frame}
% ----------------- SLIDE 15 --------------------------------
\begin{frame}{Aplicações Cliente/Servidor}
	\begin{itemize}
		\item Conjuntos de programas no cliente e servidor; \pause
		\item Solicitações partem do cliente; \pause
		\item Servidor provê o recurso; \pause
		\item Geralmente utilizam o protocolo HTTP; \pause
		\item Muito utilizados na construção de sistemas informação.
	\end{itemize}
\end{frame}

% ----------------- SLIDE 16 --------------------------------

\begin{frame}{Protocolo}
	\begin{block}{Definição de Protocolo}
		\begin{enumerate}
			\item Ata, nota ou registro dos documentos governamentais, dos 
			atos oficiais, da correspondência de um governo ou tribunal, de uma 
			empresa, universidade etc.
			\item Subdivisão de uma repartição pública (ou empresa privada) em 
			que se registram e/ou se recebem os requerimentos, documentos ou 
			processos.
			\item Recibo que registra o número e a data em que um processo ou 
			requerimento foi catalogado e registrado.
			\item Acordo regulamentado entre países ou empresas: protocolo 
			internacional.
		\end{enumerate}
	\end{block}
\end{frame}


\begin{frame}{Protocolo de Comunicação}
\begin{itemize}
\item Convenção; \pause
\item Controla e possibilita conexão, comunicação e transferência; \pause
\item Sintaxe: Inclui elementos como formato de dados e níveis de sinal; \pause
\item Semântica: Inclui informações de controle para coordenação e 
tratamento de erro; \pause
\item Temporização: Inclui combinação de velocidade e sequência. \pause
\end{itemize}
\end{frame}

% ----------------- SLIDE 18 --------------------------------
\begin{frame}{TCP/IP}
	\begin{block}{}
		\begin{itemize}
			\item Não é um modelo; \pause
			\item Conjunto de protocolos; \pause
			\item Organizados em camadas; \pause
			\item Cinco camadas.
		\end{itemize}
	\end{block}
\end{frame}
% ----------------- SLIDE 19 --------------------------------
\begin{frame}
	\begin{block}{Camadas TCP/IP}
			\centering
			\begin{table}
				\begin{tabular}{|c|}
					\hline
					Aplicação \\ \hline
					Transporte \\ \hline
					Rede \\ \hline
					Enlace\\ \hline
					Física\\ \hline
				\end{tabular}
			\end{table}
	\end{block}
\end{frame}


% ----------------- SLIDE 20 --------------------------------
\begin{frame}{HTTP}
	\begin{itemize}
		\item Protocolo básico da WWW; \pause
		\item Camada de aplicação;
		\item Pode ser usada em qualquer aplicação cliente/servidor.
		\item Transferir texto, hipertexto, áudio, imagem, etc.
	\end{itemize}
\end{frame}
