\begin{resumo}[Abstract]
 \begin{otherlanguage*}{english}
The goal of this work is to identify the Nginx HTTP server efficiency, for 
processing requisitions compared to the Apache HTTP server, currently in use by 
the Federal University of the Vales of Jequitinhonha and Mucuri – UFVJM in its 
Academic Integrated Management System - SIGA. Both server studied are among the 
seven most used in the world and among the five most used by the most accessed 
web sites wide world. The data has been collected after performance tests 
performed with the ApacheBench tool, installed in two virtual machines, equally 
configured and installed in the visualization software VirtualBox hosted in a 
portable computer.\\
Based on the amount of users in the academic community oh the UFVJM, was 
defined fifteen tests values, four simultaneous access scenarios and six 
efficiency metrics. In order to provide a better visualization of the results 
acquired, the collected data was distributed in tree graphs divided in tree 
ranges of values an a graph with the total amount of requisitions tested. The 
result analysis was made for each metric in an individual way, showing the 
average performance from each server. After the analysis, the conclusion about 
the Nginx and Apache HTTP server is shown and concludes about which server is 
more adequate e efficient to SIGA.
   \vspace{\onelineskip}
 
   \noindent 
   \textbf{Key-words}: Apache. Nginx. SIGA. UFVJM.
 \end{otherlanguage*}
\end{resumo}