\setlength{\absparsep}{18pt} % ajusta o espaçamento dos parágrafos do resumo
\begin{resumo}
O presente trabalho tem por objetivo identificar o grau de eficiência do 
servidor HTTP Nginx, para processamento das requisições recebidas, em 
comparação ao servidor Apache HTTP Server, atualmente utilizado pela 
Universidade Federal dos Vales do Jequitinhonha e Mucuri – UFVJM em seu Sistema 
Integrado de Gestão Acadêmica - SIGA. Ambos os servidores em estudo estão entre 
os sete mais utilizados no mundo e entre os cinco mais utilizados pelos sítios 
mais acessados mundialmente. Os dados foram coletados após realização de testes 
de desempenho com o uso da ferramenta ApacheBench, instalada em duas máquinas 
virtuais, igualmente configuradas e instaladas através do software de 
virtualização VirtualBox, hospedadas em computador portátil.\\
Com base em levantamento do número de usuários da comunidade acadêmica da 
UFVJM, foram definidos quinze valores de testes, quatro cenários  de acesso 
simultâneo e seis métricas indicadoras de eficiência. A fim de permitir uma 
melhor visualização dos resultados obtidos e análise, os dados gerados foram 
dispostos em três gráficos divididos em três faixas de valores e um gráfico com 
a quantidade total de requisições testadas. A análise dos resultados foi 
realizada para cada métrica, de modo individualizado, explicitando a média de 
desempenho de cada servidor. Após análise é exposta a conclusão acerca da 
eficiência demonstrada por pelo servidor HTTP Nginx e Apache HTTP Server e 
parecer sobre  qual servidor se mostra  mais adequado e eficiente ao SIGA.


 \textbf{Palavras-chaves}: Apache. Nginx. SIGA. UFVJM.
\end{resumo}