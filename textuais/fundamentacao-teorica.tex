\chapter{Fundamentação Teórica}\label{cap:fundamentacao-teorica}
Este capítulo se destina a apresentar alguns conceitos para o entendimento da 
proposta desse estudo.
\section{Aplicação cliente/servidor}
Para entender o que é uma aplicação cliente/servidor, irei primeiramente definir o que é cliente e servidor.
\subsection{Cliente}
Cliente é:
\begin{citacao}
Um solicitante de informações em rede, normalmente um PC ou estação de trabalho, que pode consultar o banco de dados e/ou outras informações de um servidor. \cite{stallings2005}
\end{citacao}
Ainda de acordo com \citeonline{stallings2005}, o cliente é um computador ou estação de trabalho, que apresenta ao usuário final uma interface gráfica de alto nível e amigável, incluindo o uso de janelas e \textit{mouse}. 
\subsection{Servidor}
Servidor é:
\begin{citacao}
Um computador, normalmente uma estação de trabalho poderosa ou um \textit{mainframe} que abriga informações para manipulação por clientes em rede. \cite{stallings2005}
\end{citacao}
Ainda de acordo com \citeonline{stallings2005}, cada servidor no ambiente cliente/servidor fornece um conjunto de serviços compartilhados.
\subsection{Aplicações cliente/servidor}
As aplicações cliente/servidor são um conjunto de programas que funcionam no computador cliente, ou seja, aquele que esta solicitando um recurso pela rede, e no computador servidor, aquele que provê o recurso solicitado. As solicitações de recursos ou informações partem do programa instalado no computador do usuário (cliente), trafegam por uma rede de computadores, geralmente utilizando o protocolo de comunicação HTTP até o servidor responsável por prover o o recurso ou informação solicitado pelo programa do cliente.\\
As aplicações cliente/servidor são muito utilizadas na construção de sistemas de informação, onde os dados armazenados pelos servidores precisam estar disponíveis para os usuários do sistema. É também o principal modelo utilizado em sistemas de informação baseados na \textit{web}, onde o cliente acessa o sistema através de um navegador \textit{web}.
\section{Protocolo de comunicação}
De acordo com o \citeonline{dicionario}, um dos significados de protocolo é: 
``Acordo 
regulamentado entre países ou empresas''. Aplicando esse conceito na 
computação, protocolo pode ser entendido como uma convenção que controla e 
possibilita uma conexão, comunicação, transferência de dados entre dois 
sistemas computacionais. De maneira simples, um protocolo pode ser definido 
como as regras que governam a sintaxe, semântica e sincronização da 
comunicação.\\
De acordo com \citeonline{stallings2005}, um protocolo é usado para a 
comunicação entre entidades em sistemas diferentes. Para que entidades 
diferentes se comuniquem com facilidade, elas precisam adotar uma ``língua'' em 
comum. O que e como é comunicado deve estar de acordo com as convenções 
combinadas entre as entidades envolvidas. Essas convenções são denominadas 
protocolos, que nada mais é do que um conjunto de regras controlando a troca de 
dados entre entidades. Os principais elementos de um protocolo são:
\begin{itemize}
	\item Sintaxe: Inclui elementos como formato de dados e níveis de sinal;
	\item Semântica: Inclui informações de controle para coordenação e 
	tratamento de erro;
	\item Temporização: Inclui combinação de velocidade e sequência.
\end{itemize}
\section{Modelo TCP/IP}
O TCP/IP (\textit{Transmission Control Protocol/Internet Protocol}), é um conjunto de protocolos, resultante de pesquisas e desenvolvimentos realizados pela ARPANET (\textit{Advanced Research Project Agency Network}) e financiada pela DARPA (\textit{Defense Advanced Research Projects Agency}), agência de pesquisa militar do governo dos Estados Unidos, e que hoje são utilizados como padrões de comunicação na Internet.\\
Ao contrário do modelo OSI(Open Systems Interconnections), o TCP/IP não pode ser considerado um modelo e sim um conjunto de protocolos. Esse protocolos, assim como no modelo OSI, podem ser organizados camadas relativamente independentes, com a diferença de que no modelo OSI existem sete camadas e no TPC/IP existem cinco, camadas essas relativamente independentes. São elas:
\begin{itemize}
	\item Camada de aplicação
	\item Camada de transporte
	\item Camada de rede
	\item Camada de enlace
	\item Camada física
\end{itemize}
Só para efeito de comparação, as camadas do modelo OSI são:
\begin{itemize}
	\item Camada de aplicação
	\item Camada de apresentação
	\item Camada de sessão
	\item Camada de transporte
	\item Camada de rede
	\item Camada de enlace
	\item Camada física
\end{itemize}
%aplicação
\subsection{Camada de aplicação}
A camada de aplicação é a camada que a maioria dos programas de rede usa de 
forma a se comunicar através de uma rede com outros programas. Processos que 
rodam nessa camada são específicos da aplicação; o dado é passado do programa 
de rede, no formato usado internamente por essa aplicação, e é codificado 
dentro do padrão de um protocolo. Existem diversos protocolos nesta camada. Os 
mais utilizados são:
\begin{itemize}
	\item SMTP (\textit{Simple Mail Transport Protocol}): é utilizado para a 
	comunicação entre serviços de correio eletrônico na Internet.
	\item POP (\textit{Post Office Protocol}): é utilizado para recuperação de 
	mensagens de correio eletrônico via Internet.
	\item IMAP (\textit{Internet Mail Access Protocol}): é utilizado para 
	recuperação de mensagens de correio eletrônico via Internet, porém é mais 
	avançado que o POP.
	\item HTTP (\textit{Hypertext Transport Protocol}): utilizado para a 
	publicação de sites \textit{web} na Internet.
	\item FTP (\textit{File Transfer Protocol}): utilizado para publicação e 
	transferência de arquivos via Internet.
\end{itemize}
%host a host
\subsection{Camada de transporte}
A camada de transporte (ou \textit{host} a \textit{host}) é responsável pela 
transferência eficiente, confiável e econômica dos dados entre o computador de 
origem e de destino, independente do tipo, topologia ou configuração das redes 
físicas existentes entre elas, garantindo ainda que os dados cheguem sem erros 
e na sequência correta.\\
A camada de transporte é uma camada fim-a-fim, isto é, uma entidade desta 
camada só se comunica com a sua entidade semelhante do destinatário. A camada 
de transporte provê mecanismos que possibilitam a troca de dados fim-a-fim, não 
possibilitando, assim, a comunicação com entidades intermediárias.\\
Esta camada reúne os protocolos que realizam as funções de transporte de dados 
fim-a-fim, ou seja, considerando apenas a origem e o destino da comunicação, 
sem se preocupar com os elementos intermediários. A camada de transporte possui 
dois protocolos que são o UDP (\textit{User Datagram Protocol}) e o TCP 
(\textit{Transmission Control Protocol}).\\
O protocolo UDP realiza apenas a multiplexação para que várias aplicações 
possam acessar o sistema de comunicação de forma coerente.\\
O protocolo TCP realiza, além da multiplexação, uma série de funções para 
tornar a comunicação entre origem e destino mais confiável. São 
responsabilidades do protocolo TCP: o controle de fluxo, o controle de erro, a 
sequenciação e a multiplexação de mensagens.\\
%inter-rede
\subsection{Camada de rede}
A camada de rede (ou inter-rede) é responsável por controlar a operação da rede 
de um modo geral. Suas principais funções são o roteamento dos pacotes entre 
fonte e destino, mesmo que estes tenham que passar por diversos nós 
intermediários durante o percurso, o controle de congestionamento e a 
contabilização do número de pacotes ou bytes utilizados pelo usuário, para fins 
de tarifação.\\
O principal aspecto que deve ser observado nessa camada é a execução do 
roteamento dos pacotes entre fonte e destino, principalmente quando existem 
caminhos diferentes para conectar entre si dois nós da rede. Em redes de longa 
distância é comum que a mensagem chegue do nó fonte ao nó destino passando por 
diversos nós intermediários no meio do caminho e é tarefa dos protocolos da 
camada de rede escolher o melhor caminho para essa mensagem.\\
A escolha da melhor rota pode ser baseada em tabelas estáticas, que são 
configuradas na criação da rede e são raramente modificadas; pode também ser 
determinada no início de cada conversação, ou ser altamente dinâmica, sendo 
determinada a cada novo pacote, a fim de refletir exatamente a carga da rede 
naquele instante. Se muitos pacotes estão sendo transmitidos através dos mesmos 
caminhos, eles vão diminuir o desempenho global da rede, formando gargalos. O 
controle de tais congestionamentos também é tarefa da camada de rede.\\
%acesso à rede
\subsection{Camada de enlace}
A camada de enlace (ou de acesso à rede) trata da troca de dados entre um 
sistema final e a rede à qual está conectada \cite{stallings2005}. Para que 
haja uma comunicação, o computador de origem da mensagem deve informar para a 
camada de enlace qual o endereço do computador de destino, para que a rede 
possa entregar a mensagem para o destinatário correto. O computador de origem 
pode requerer alguns serviços especiais, como por exemplo, prioridade no envio 
da mensagem. O software utilizado nessa camada irá depender do tipo de rede a 
ser usado, sendo que diferentes padrões foram desenvolvidos para comutação de 
circuitos pacotes.\\
Quando dois computadores estão conectado a redes diferentes, se faz necessário 
o uso de procedimentos para que os dados trafeguem entre as redes. Nesses 
casos, se utiliza o protocolo IP (\textit{Internet Protocol}). Esse protocolo 
oferece a função de interconectar várias redes através de roteadores. Os 
roteadores são computadores especializados em conectar duas ou mai redes 
diferentes, e o protocolo IP é também utilizado e implementado nos roteadores.
\subsection{Camada física}
A camada de interface de rede ou física é a primeira camada. Também chamada 
camada de abstração de hardware, tem como função principal a interface do 
modelo TCP/IP com os diversos tipos de redes (X.25, ATM, FDDI, Ethernet, 
\textit{Token Ring}, \textit{Frame Relay}, sistema de conexão ponto-a-ponto 
SLIP, etc.) e transmitir os datagramas pelo meio físico.\\
Esta camada lida com os meios de comunicação, corresponde ao nível de hardware, 
ou meio físico, que trata dos sinais eletrônicos, conector, pinagem, níveis de 
tensão, dimensões físicas, características mecânicas e elétricas etc. Os 
protocolos da camada física enviam e recebem dados em forma de pacotes, que 
contém um endereço de origem, os dados propriamente ditos e um endereço de 
destino. Os datagramas já foram construídos pela camada de rede.\\
É responsável pelo endereçamento e tradução de nomes e endereços lógicos em 
endereços físicos. Ela determina a rota que os dados seguirão do computador de 
origem até o de destino. Tal rota dependerá das condições da rede, prioridade 
do serviço dentre outros fatores.\\
Também gerencia o tráfego e taxas de velocidade nos canais de comunicação.Outra 
função que pode ter é o agrupamento de pequenos pacotes em um único para  
transmissão pela rede (ou a subdivisão de pacotes grandes). No destino os dados 
são recompostos no seu formato original.\\
\section{\textit{World Wide Web}}
A \textit{World Wide Web} (ou WWW, ou \textit{Web}), foi proposta em 1.989 pelo 
cientista britânico Sir Tim Berners-Lee quando trabalhava no CERN (Laboratório 
Europeu para Partículas Físicas). A ideia de Berners-Lee era propor uma 
tecnologia de hipermídia distribuída para o compartilhamento internacional de 
descobertas científicas usando a Internet.\\
A \textit{web} é um sistema distribuído, que consiste em uma coleção de 
arquivos armazenados em servidores e que podem ser acessados a partir de 
programas instalados nos computadores dos clientes chamados navegadores. Cada 
arquivo tem um endereço em forma de URL. Os usuário podem navegar de um arquivo 
para outro (ou entre páginas) fazendo uso do \textit{mouse} do computador para 
clicar em um \textit{link} que irá direcionar o usuário para a página 
requisitada. Por ser uma tecnologia que está presente na camada de aplicação, o 
protocolo utilizado é o HTTP\\
\section{Protocolo HTTP}
De acordo com \citeonline{stallings2005}, o \textit{Hypertext Transfer 
Protocol} (Protocolo de Transferência de Hipertexto em tradução livre) é o 
protocolo básico da \textit{World Wide Web} e pode ser usado em qualquer 
aplicação cliente/servidor que envolve hipertexto e faz parte da camada de 
aplicação do modelo TCP/IP.\\
O HTTP não serve somente para transferir hipertexto. É um protocolo para transferir informações com uma eficiência necessária. Os dados transferidos podem ser texto puro, hipertexto, áudio, imagens ou qualquer outro dado disponível na Internet.\\
O HTTP é um protocolo cliente/servidor orientado para transações. O seu uso 
mais típico é acontece entre um navegador \textit{web} e um servidor 
\textit{web}, sempre utilizando o protocolo da camada de transporte TCP afim de 
manter a confiabilidade.\\
Cada transação utiliza uma nova conexão TCP entre o cliente e o servidor,sendo 
tratada de forma independente. Após o término de cada transação, a conexão 
entre cliente e servidor é fechada. Por ser um protocolo sem estado, é o 
adequado para a maior parte das aplicações.\\
\section{\textit{Framework}}

\section{Servidor \textit{web}}
De acordo com \citeonline{laurieandlaurie2003}, a principal função de um 
servidor web é traduzir uma URL em um arquivo, que será enviado de volta pela 
Internet, ou em um programa que será executado e o resultado será enviado de 
volta.\\
Quando o usuário abre o navegador e digita uma URL, é enviada uma 
mensagem pela internet para o computador que atende aquele endereço.\\
URL significa Localizador Uniforme de Recurso e pode ser separada em três 
partes:
\begin{verbatim}
	<protocolo>://<destinatário>/<caminho>
\end{verbatim}
\section{C10K \textit{problem}}
O C10k \textit{problem} consiste em solucionar o problema de atender a 10.000 requisições concorrentemente.\\
\textbf{Escrever sobre}