\chapter{Fundamentação Teórica}\label{cap:fundamentacao-teorica}
Este capítulo se destina a apresentar alguns conceitos para o desenvolvimento da proposta de alteração de servidor, apresentando alguns conceitos sobre redes e comunicação de computadores.
\section{Aplicação cliente/servidor}
Para entender o que é uma aplicação cliente/servidor, irei primeiramente definir o que é cliente e servidor.
\subsection{Cliente}
Cliente é:
\begin{citacao}
Um solicitante de informações em rede, normalmente um PC ou estação de trabalho, que pode consultar o banco de dados e/ou outras informações de um servidor. \cite{stallings2005}
\end{citacao}
Ainda de acordo com \citeonline{stallings2005}, o cliente é um computador ou estação de trabalho, que apresenta ao usuário final uma interface gráfica de alto nível e amigável, incluindo o uso de janelas e \textit{mouse}. 
\subsection{Servidor}
Servidor é:
\begin{citacao}
Um computador, normalmente uma estação de trabalho poderosa ou um \textit{mainframe} que abriga informações para manipulação por clientes em rede. \cite{stallings2005}
\end{citacao}
Ainda de acordo com \citeonline{stallings2005}, cada servidor no ambiente cliente/servidor fornece um conjunto de serviços compartilhados.
\subsection{Aplicações cliente/servidor}
As aplicações cliente/servidor são um conjunto de programas que funcionam no computador cliente, ou seja, aquele que esta solicitando um recurso pela rede, e no computador servidor, aquele que provê o recurso solicitado. As solicitações de recursos ou informações partem do programa instalado no computador do usuário (cliente), trafegam por uma rede de computadores, geralmente utilizando o protocolo de comunicação HTTP até o servidor responsável por prover o o recurso ou informação solicitado pelo programa do cliente.\\
As aplicações cliente/servidor são muito utilizadas na construção de sistemas de informação, onde os dados armazenados pelos servidores precisam estar disponíveis para os usuários do sistema. É também o principal modelo utilizado em sistemas de informação baseados na \textit{web}, onde o cliente acessa o sistema através de um navegador \textit{web}.
\section{Protocolo de comunicação}
\section{Modelo TCP/IP}
O TCP/IP (\textit{Transmission Control Protocol/Internet Protocol}), é um conjunto de protocolos, resultante de pesquisas e desenvolvimentos realizados pela ARPANET (\textit{Advanced Research Project Agency Network}) e financiada pela DARPA (\textit{Defense Advanced Research Projects Agency}), agência de pesquisa militar do governo dos Estados Unidos, e que hoje são utilizados como padrões de comunicação na Internet.\\
Ao contrário do modelo OSI(Open Systems Interconnections), o TCP/IP não pode ser considerado um modelo e sim um conjunto de protocolos. Esse protocolos, assim como no modelo OSI, podem ser organizados camadas relativamente independentes, com a diferença de que no modelo OSI existem sete camadas e no TPC/IP existem cinco, camadas essas relativamente independentes. São elas:
\begin{itemize}
	\item Camada de aplicação
	\item Camada de transporte
	\item Camada de rede
	\item Camada de enlace
	\item Camada física
\end{itemize}
Só para efeito de comparação, as camadas do modelo OSI são:
\begin{itemize}
	\item Camada de aplicação
	\item Camada de apresentação
	\item Camada de sessão
	\item Camada de transporte
	\item Camada de rede
	\item Camada de enlace
	\item Camada física
\end{itemize}
\subsection{Camada de aplicação e seus principais protocolos}
\subsection{Camada de transporte e seus principais protocolos}
\subsection{Camada de rede e seus principais protocolos}
\subsection{Camada de enlace e seus principais protocolos}
\subsection{Camada de física e seus principais protocolos}
\section{Protocolo TCP}
\section{\textit{World Wide Web}}
\section{Protocolo HTTP}
De acordo com \citeonline{stallings2005}, o \textit{Hypertext Transfer Protocol} (Protocolo de Transferência de Hipertexto em tradução livre) é o protocolo básico da \textit{World Wide Web} e pode ser usado em qualquer aplicação cliente/servidor que envolve hipertexto e faz parte da camada de transporte do modelo TCP/IP.\\
O HTTP não serve somente para transferir hipertexto. É um protocolo para transferir informações com uma eficiência necessária. Os dados transferidos podem ser texto puro, hipertexto, áudio, imagens ou qualquer outro dado disponível na Internet.\\
O HTTP é um protocolo cliente/servidor orientado para transações. O seu uso mais típico é acontece entre um navegador \textit{web} e um servidor \textit{web}, sempre utilizando o modelo TCP afim de manter a confiabilidade.\\
Cada transação utiliza uma nova conexão TCP entre o cliente e o servidor, sendo cada transação tratada de forma independente. Após o término de cada transação, a conexão entre cliente e servidor é fechada. Por ser um protocolo sem estado, é o protocolo adequado para a maior parte das aplicações.
\section{\textit{Framework}}
\section{Servidor \textit{web}}
\section{C10K \textit{problem}}
O C10k \textit{problem} consiste em solucionar o problema de atender a 10.000 requisições concorrentemente.\\
\textbf{Escrever sobre}