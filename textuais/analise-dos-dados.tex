\chapter{Análise dos dados}\label{cap:analise-dos-dados}
A análise dos dados é essencial para este estudo. É a partir dessa análise que 
será possível determinar se o servidor HTTP objeto do estudo, o Nginx é 
realmente mais eficiente do que o Apache. Os dados utilizados para gerar dos 
gráficos são a média dos quatro testes realizados para cada quantidade total de 
requisições.\\
A análise será feita a partir da construção de gráficos comparativos entre os 
servidores Apache e Nginx, analisando cada métrica disponibilizada pelo 
ApacheBench.
\section{Organização dos gráficos}
Para melhor analisar o comportamento dos dois servidores em cada métrica, foram 
gerados quatro gráficos: três separados em faixas de 
acordo com a quantidade total de requisições e um gráfico com todas as 
quantidades totais de requisições testadas. As faixas de valores são:
\begin{itemize}
	\item \textbf{Faixa 1} - Entre 1.000 e 5.000 requisições totais;
	\item \textbf{Faixa 2} - Entre 6.000 e 10.000 requisições totais;
	\item \textbf{Faixa 3} - Entre 11.000 e 15.000 requisições totais;
\end{itemize}
Dispondo os gráficos dessa forma, será possível analisar os dados de acordo com 
a quantidade de usuários utilizando o sistema e analisar o comportamento do 
servidor HTTP de forma geral. Os gráficos serão apresentados todos de uma só 
vez e a análise dos dados será feita na seção \ref{sec:analise-dos-dados}.
\section{Gráficos}
Nessa seção será apresentado todos os gráficos gerados a partir dos dados 
coletados. A análise dos dados será feita na seção 
\ref{sec:analise-dos-dados}.\\
Todos os gráficos foram gerados usando a ferramenta de geração de gráfico do 
\citeonline{LOcalc}.
% %GRAFICO 1
\subsection{Média do Tempo Total de Execução Dos Testes}

\begin{figure}[H]
	\centering
	\includegraphics[width=1\linewidth]{graficos/grafico1-f1} 
	\caption{Média do Tempo Total de Execução dos Testes - Faixa 1}
	\label{fig:grafico1-f1}
\end{figure}

\begin{figure}[H]
	\centering
	\includegraphics[width=1\linewidth]{graficos/grafico1-f2} 
	\caption{Média do Tempo Total de Execução dos Testes - Faixa 2}
	\label{fig:grafico1-f2}
\end{figure}

\begin{figure}[H]
	\centering
	\includegraphics[width=1\linewidth]{graficos/grafico1-f3} 
	\caption{Média do Tempo Total de Execução dos Testes - Faixa 3}
	\label{fig:grafico1-f3}
\end{figure}

\begin{figure}[H]
	\centering
	\includegraphics[width=1\linewidth]{graficos/grafico1} 
	\caption{Média do Tempo Total de Execução dos Testes}
	\label{fig:grafico1}
\end{figure}

% %GRAFICO 2
\subsection{Média do Total de Dados Transferido}
\begin{figure}[H]
	\centering
	\includegraphics[width=1\linewidth]{graficos/grafico2-f1} 
	\caption{Média do Total de Dados Transferido - Faixa 1}
	\label{fig:grafico2-f1}
\end{figure}

\begin{figure}[H]
	\centering
	\includegraphics[width=1\linewidth]{graficos/grafico2-f2} 
	\caption{Média do Total de Dados Transferido - Faixa 2}
	\label{fig:grafico2-f2}
\end{figure}

\begin{figure}[H]
	\centering
	\includegraphics[width=1\linewidth]{graficos/grafico2-f3} 
	\caption{Média do Total de Dados Transferido - Faixa 3}
	\label{fig:grafico2-f3}
\end{figure}

\begin{figure}[H]
	\centering
	\includegraphics[width=1\linewidth]{graficos/grafico2} 
	\caption{Média do Total de Dados Transferido}
	\label{fig:grafico2}
\end{figure}

% %GRAFICO 3
\subsection{Média do Total de Texto HTML Transferido}
\begin{figure}[H]
	\centering
	\includegraphics[width=1\linewidth]{graficos/grafico3-f1} 
	\caption{Média do Total de Texto HTML Transferido - Faixa 1}
	\label{fig:grafico3-f1}
\end{figure}

\begin{figure}[H]
	\centering
	\includegraphics[width=1\linewidth]{graficos/grafico3-f2} 
	\caption{Média do Total de Texto HTML Transferido - Faixa 2}
	\label{fig:grafico3-f2}
\end{figure}

\begin{figure}[H]
	\centering
	\includegraphics[width=1\linewidth]{graficos/grafico3-f3} 
	\caption{Média do Total de Texto HTML Transferido - Faixa 3}
	\label{fig:grafico3-f3}
\end{figure}

\begin{figure}[H]
	\centering
	\includegraphics[width=1\linewidth]{graficos/grafico3} 
	\caption{Média do Total de Texto HTML Transferido}
	\label{fig:grafico3}
\end{figure}

% %GRAFICO 4
\subsection{Média do Número de Requisições Atendidas}
\begin{figure}[H]
	\centering
	\includegraphics[width=1\linewidth]{graficos/grafico4-f1} 
	\caption{Média do Número de Requisições Atendidas - Faixa 1}
	\label{fig:grafico4-f1}
\end{figure}

\begin{figure}[H]
	\centering
	\includegraphics[width=1\linewidth]{graficos/grafico4-f2} 
	\caption{Média do Número de Requisições Atendidas - Faixa 2}
	\label{fig:grafico4-f2}
\end{figure}

\begin{figure}[H]
	\centering
	\includegraphics[width=1\linewidth]{graficos/grafico4-f3} 
	\caption{Média do Número de Requisições Atendidas - Faixa 3}
	\label{fig:grafico4-f3}
\end{figure}

\begin{figure}[H]
	\centering
	\includegraphics[width=1\linewidth]{graficos/grafico4} 
	\caption{Média do Número de Requisições Atendidas}
	\label{fig:grafico4}
\end{figure}


% %GRAFICO 5
\subsection{Média do Tempo de Resposta por Requisição Simultânea}
\begin{figure}[H]
	\centering
	\includegraphics[width=1\linewidth]{graficos/grafico5-f1} 
	\caption{Média do Tempo de Resposta por Requisição Simultânea - Faixa 1}
	\label{fig:grafico5-f1}
\end{figure}

\begin{figure}[H]
	\centering
	\includegraphics[width=1\linewidth]{graficos/grafico5-f2} 
	\caption{Média do Tempo de Resposta por Requisição Simultânea - Faixa 2}
	\label{fig:grafico5-f2}
\end{figure}

\begin{figure}[H]
	\centering
	\includegraphics[width=1\linewidth]{graficos/grafico5-f3} 
	\caption{Média do Tempo de Resposta por Requisição Simultânea - Faixa 3}
	\label{fig:grafico5-f3}
\end{figure}

\begin{figure}[H]
	\centering
	\includegraphics[width=1\linewidth]{graficos/grafico5} 
	\caption{Média do Tempo de Resposta por Requisição Simultânea}
	\label{fig:grafico5}
\end{figure}



% %GRAFICO 6
\subsection{Média da Taxa de Transferência}
\begin{figure}[H]
	\centering
	\includegraphics[width=1\linewidth]{graficos/grafico6-f1} 
	\caption{Média da Taxa de Transferência - Faixa 1}
	\label{fig:grafico6-f1}
\end{figure}

\begin{figure}[H]
	\centering
	\includegraphics[width=1\linewidth]{graficos/grafico6-f2} 
	\caption{Média da Taxa de Transferência - Faixa 2}
	\label{fig:grafico6-f2}
\end{figure}

\begin{figure}[H]
	\centering
	\includegraphics[width=1\linewidth]{graficos/grafico6-f3} 
	\caption{Média da Taxa de Transferência - Faixa 3}
	\label{fig:grafico6-f3}
\end{figure}

\begin{figure}[H]
	\centering
	\includegraphics[width=1\linewidth]{graficos/grafico6} 
	\caption{Média da Taxa de Transferência}
	\label{fig:grafico6}
\end{figure}




\begin{comment}
% %GRAFICO 3
\subsection{3}
\begin{figure}[htb]
	\centering
	\includegraphics[width=1\linewidth]{graficos/grafico3-f1} 
	\caption{ - Faixa 1}
	\label{fig:grafico-f1}
\end{figure}

\begin{figure}[htb]
	\centering
	\includegraphics[width=1\linewidth]{graficos/grafico3-f2} 
	\caption{ - Faixa 2}
	\label{fig:grafico-f2}
\end{figure}

\begin{figure}[htb]
	\centering
	\includegraphics[width=1\linewidth]{graficos/grafico3-f3} 
	\caption{ - Faixa 3}
	\label{fig:grafico-f3}
\end{figure}

\begin{figure}[htb]
	\centering
	\includegraphics[width=1\linewidth]{graficos/grafico3} 
	\caption{}
	\label{fig:grafico}
\end{figure}
\end{comment}





\section{Análise dos Dados}\label{sec:analise-dos-dados}
