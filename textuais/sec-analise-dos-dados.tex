\section{Análise}\label{sec:analise-dos-dados}
Os dados serão analisados métrica à métrica de acordo com a análise dos 
gráficos e dos dados coletados.
\subsection{Tempo Total de Execução do Teste}
O tempo total de execução do tem como finalidade mostrar o tempo em que o 
processo do ApacheBench responsável pelo teste ficou em execução no sistema 
operacional. Tendo em vista que o processo do ApacheBench pode fica em execução 
por mais tempo do que a realização do teste e que uma requisição pode demorar a 
responder, atrasando o fim do teste, essa métrica dá uma noção do desempenho do 
servidos, mais não deve ser levado em consideração em uma análise adequada.\\
Ao analisar o gráfico e os dados da Faixa 1, é possível ver que os testes no 
servidor Apache demoraram, em média 5,85 vezes mais do que no servidor Nginx. 
Na Faixa 2, os testes demoraram 6,95 vezes mais para serem executados no 
Apache. Na Faixa 3, os testes no Nginx demoraram, em média 5,64 vezes menos 
para serem executados. O pior caso em termos de diferença de desempenho 
aconteceu nos testes com 6.000 requisições totais, onde o Apache demorou 
aproximadamente 7,11 vezes mais para seres executados (20,61s no Nginx contra 
146,64 segundos no Apache).\\
Em termos gerais, levando em consideração os quinze valores de requisições 
totais, o tempo de execução dos testes no Apache foram em média 6,76 vezes mais 
lentos do que no Nginx. A soma dos tempos de totais de teste foram: 439,44 
segundos para o Nginx contra 2522,08 segundos para o Apache.
\subsection{Total de Dados Transferidos}
O total de dados transferidos mostra a quantidade de informações que foram 
trocadas entre os computadores, incluindo os dados enviados pelas requisições e 
os dados retornados pelo servidor.\\
Os dados e os gráficos mostram que na média, os testes no servidor do Nginx 
transferiram 4,3 vezes mais dados do que nos testes no Apache. 

