\section{Ambiente de testes}

Para fazer os teste da forma mais neutra possível, foram criadas duas máquinas virtuais usando o software de virtualização VirtualBox na sua versão 3.4.10.\\
Todos os programas utilizados nos ambientes de teste foram instalados pelo gerenciador de pacotes nativo da distribuição Linux Debian \textit{aptitude} e disponíveis nos repositórios oficiais do Debian, exceto o SGBD PostgreSQL que foi instalado usando o \textit{aptitude} porém através do repositório oficial do PostgreSQL para o Linux Debian 7 pois nos repositórios do Debian não havia disponível a versão mais recente e utilizada pelo banco de dados do SIGA.

\subsection{Computador hospedeiro}
O computador onde foram instaladas as máquinas virtuais possui a seguinte 
configuração:
\begin{itemize}
	\item \textbf{Tipo}: Computador portátil (\textit{Notebook})
	\item \textbf{Processador}: Intel Core i5-3337U
	\item \textbf{Frequência do Processador}: 1,8 Giga Hertz
	\item \textbf{Tamanho da Memória Principal (RAM)}: 8 Giga Bytes;
	\item \textbf{Tamanho da Memória Secundária (HD)}: 1 Tera Bytes;
	\item \textbf{Quantidade de núcleos disponível para processamento}: 4 
	núcleos.
\end{itemize}

\subsection{Configurações comuns às duas máquinas virtuais}
 Ambas máquinas virtuais tinhas as configurações rigorosamente iguais. São elas:
\begin{itemize}
\item \textbf{Sistema Operacional}: Linux Debian 7 “Wheezy” 64 bits mais atualizado;
\item \textbf{Sistema Gerenciador de Banco de Dados}: PostgreSQL na versão 9.3.5
\item \textbf{Tamanho da Memória Principal (RAM)}: 2.048 Mega Bytes;
\item \textbf{Tamanho da Memória Secundária (HD)}: 30 Giga Bytes;
\item \textbf{Quantidade de núcleos disponível para processamento}: 1 núcleo.
\end{itemize}

\subsection{Máquina virtual Apache}
Na máquina virtual destinada aos testes com o servidor HTTP Apache, os programas utilizados foram:
\begin{itemize}
\item Apache HTTP Server na versão 2.2.22;
\item libapache2-mod-php5 na versão 5.4.4;
\item php5-common na versão 5.4.4.
\end{itemize}

\subsection{Máquina virtual Nginx}
Na maquina virtual destinada aos testes com o servidor HTTP Nginx, os programas utilizados foram:

\begin{itemize}
\item Servidor HTTP Nginx na versão 1.2.1;
\item php5-fpm na versão 5.4.4;
\item php5-common na versão 5.4.4.
\end{itemize}

