\chapter{Introdução}\label{introducao}
Com o avanço dos computadores e das redes de comunicação, os sistemas de informação foram sendo transferidos dos computadores pessoais para os servidores \textit{web}, de onde eles podem ser acessados, virtualmente, a partir de todo o mundo. Para que esses sistemas funcionem, eles precisam de \textit{softwares} que atendam as requisições que chegam ao servidor a partir da rede de computadores usando o protocolo HTTP. Esses softwares são chamados Servidores HTTP.\\
Hoje o SIGA – Sistema Integrado de Gestão Acadêmica, desenvolvido utilizando a linguagem de programação PHP e o framework Miolo, utiliza o Apache HTTP \textit{Server} como servidor HTTP.\\
\section{Objetivos}
O objetivo desse estudo é identificar se a utilização do servidor HTTP NGINX é mais eficiente do que o que é utilizado atualmente, o APACHE HTTP \textit{Server}.\\
\section{Motivação}
Com a expansão das Universidades Federais, por ano são admitidos 3500 alunos na graduação, pós-graduação e graduação e especialização à distância, além de novos servidores técnicos-administrativos e professores. No final de 2.013, data do último levantamento, a Universidade Federal dos Vales do Jequitinhonha e Mucuri tinha 8.121 alunos, 576 professores e 421 servidores técnicos-administrativos, espalhados em quatro \textit{campi} universitários, fazendas experimentais e pólos de ensino em educação à distância, totalizando 9.118 pessoas que interagem com a universidade diariamente.\\
Em épocas de pico de utilização do SIGA, as reclamações de lentidão e problemas no sistemas são frequentes, as vezes impossibilitando a utilização do mesmo. Os picos mais notório são: fim de período letivo da graduação, quando alunos e professores acessam o sistema para olhar e lançar notas, respectivamente; e rematricula dos alunos da graduação, quando os mesmos escolhem as matérias que desejam cursar no período seguinte.\\
Com o crescente aumento de alunos, servidores públicos (professores e técnicos-administrativos) e teceirizados na universidade, a tendência é que a utilização do SIGA se torne mais problemática.
Com isso em mente, a utilização do servidor HTTP NGINX pode ajudar a amenizar os problema de desempenho do SIGA.\\
