\chapter{Metodologia}\label{cap:metodologia}
Para comparar o desempenho do Apache e do Nginx, foram realizados testes usando 
a ferramenta ApacheBench.\\
O ApacheBench é uma ferramenta para ser utilizada em linha de comando. Para 
invocar a ferramenta escreve-se o comando ``ab'' no terminal. A ferramenta 
aceita como parâmetros, entre outros, a quantidade de requisições que serão 
feitas, quantas requisições serão feitas de forma simultânea e a URL da página 
que será requisitada. O comando de uma forma genérica fica assim:
\begin{verbatim}
ab -n numero_de_requisições -c requisições_simultâneas endereço_do_servidor
\end{verbatim}
O parâmetro ``-n'' indica a quantidade total de requisições serão feitas no 
teste e o parâmetro ``-c'' indica a quantidade de requisições serão feitas de 
forma simultânea.\\
Para comparar os desempenhos dos dois servidores HTTP, foram realizados teste 
com quantidade de requisições entre 1.000 (mil) e 15.000 (quinze mil) pulando 
de mil em mil, com 10\%, 20\%, 40\% e 80\% de concorrência de acordo com a 
quantidade de requisições. Então para cada servidor HTTP, foram feitos 75 
testes listados abaixo:
\begin{table}[htb]
	\centering
\ABNTEXfontereduzida
\caption[Requisições Totais e Requisições Concorrentes]{Requisições Totais e Requisições Concorrentes}
\label{tab:requisicoes}
\begin{tabular}{|>{\bfseries}c|c|c|c|c|}
\hline
\multirow{2}{*}{Requisições Totais} & \multicolumn{4}{c|}{\textbf{Requisições 
Concorrentes}} \\ \cline{2-5}
& 10\%      & 20\%  & 40\%  & 80\%  \\ \hline
1.000  & 100   & 200   & 400   & 800   \\ \hline
2.000  & 200   & 400   & 800   & 1.600  \\ \hline
3.000  & 300   & 600   & 1.200 & 2.400  \\ \hline
4.000  & 400   & 800   & 1.600 & 3.200  \\ \hline
5.000  & 500   & 1.000 & 2.000 & 4.000  \\ \hline
6.000  & 600   & 1.200 & 2.400 & 4.800  \\ \hline
7.000  & 700   & 1.400 & 2.800 & 5.600  \\ \hline
8.000  & 800   & 1.600 & 3.200 & 6.400  \\ \hline
9.000  & 900   & 1.800 & 3.600 & 7.200  \\ \hline
10.000 & 1.000 & 2.000 & 4.000 & 8.000  \\ \hline
11.000 & 1.100 & 2.200 & 4.400 & 8.800  \\ \hline
12.000 & 1.200 & 2.400 & 4.800 & 9.600  \\ \hline
13.000 & 1.300 & 2.600 & 5.200 & 10.400 \\ \hline
14.000 & 1.400 & 2.800 & 5.600 & 11.200 \\ \hline
15.000 & 1.500 & 3.000 & 6.000 & 12.000 \\ \hline
\end{tabular}
\legend{Quantidade de Requisições Totais e Requisições Concorrentes}
\end{table}
Para cada teste realizado os seguintes dados foram obtidos a partir da resposta do servidor HTTP para o ApacheBench:
\begin{itemize}
	\item Tempo total do teste em segundos (s);
	\item Total de dados transferido em bytes (b);
	\item Total de texto em HTML transferido em bytes (b);
	\item Número de requisições atendidas por segundo (N/s);
	\item Tempo médio por requisição em milissegundos (ms);
	\item Tempo médio por requisição entre as requisições concorrentes em milissegundos (ms);
	\item Taxa de transferência em Quilo Bytes por segundo (Kb/s);
	\item Porcentagem das requisições servidas em um período de tempo, tempo em milissegundos (X\%ms).
\end{itemize}
