\chapter{Conclusão}\label{cap:conclusao}
Com base nos dados coletados nos teste com os dois servidores, é possível dizer 
que o servidor HTTP Nginx conseguiu ser, em média, 5 vezes mais eficiente do 
que o servidor HTTP Apache, dando uma margem de sobrevida ao SIGA como ele é 
desenvolvido hoje, usando o \textit{framework} Miolo. Porém, mesmo com a 
utilização de um servidor HTTP mais eficiente, é difícil dizer até quando o  
SIGA aguentará a crescente demanda de novos usuários.\\
Como dito na introdução, escalabilidade é um atributo desejável de um sistema. 
Ao meu ver, o \textit{framework} Miolo não oferece um nível de escalabilidade 
suficientemente bom, sendo necessária a compra de novos servidores com 
desempenho superior ao que é utilizado atualmente, para que o sistema suporte 
uma grande demanda. O Miolo, ao ``esconder'' o HTML do desenvolvedor com a 
intenção de agilizar o desenvolvimento, acaba por exigir uma carga de 
processamento exagerada, devido aos seus mecanismos de geração de páginas 
dinâmicas.\\
Além da substituição do servidor HTTP para o Nginx, o ideal seria a total 
substituição do SIGA, sendo desenvolvido do zero, podendo utilizar a mesma 
linguagem de programação (PHP) porém com um \textit{framework} mais atual e em 
constante desenvolvimento, reduzindo a curva de aprendizado. Alguns dos  
\textit{frameworks} PHP mais utilizados hoje são:
\begin{itemize}
	\item Zend Framework
	\item CakePHP
	\item Code Igniter
	\item Laravel
\end{itemize}
Uma substituição mais radical envolveria a mudança até mesmo da linguagem de 
programação utilizada no desenvolvimento, sendo Ruby junto com o 
\textit{framework} Ruby on Rails, e a linguagem Python junto com o 
\textit{framework} Django, os mais recomendados, por serem utilizadas com 
sucesso por várias empresas, serem de código aberto (sem custo com licenças) e 
contam com uma grande comunidade de desenvolvedores.
\section{Trabalhos futuros}
A troca do servidor HTTP, do Apache para o Nginx, não é a única coisa que pode 
ser feita para que o SIGA não tenha o seu ciclo de vida encerrado em um futuro 
próximo.\\
A atualização dos \textit{softwares} utilizados pelo SIGA traria não somente 
uma melhoria no desempenho, como também mais segurança. A versão do PHP 
utilizada no servidor do SIGA a 5.3.3, lançada em Junho de 2.009, 5 anos atrás, 
e a mais atual é a 5.6, lançada em Outubro de 2.014. A versão do 
\textit{framework} Miolo utilizado no SIGA é a 2.0, sendo a versão mais atual a 
2.6.\\
A ferramenta de relatórios utilizada representa outro gargalo importante no 
desempenho do SIGA. O Jaspereport foi desenvolvida originalmente para ser 
utilizada com a linguagem Java, sendo que, para cada relatório gerado, uma 
máquina virtual Java é criada para cada relatório gerado, consumindo muitos 
recursos do servidor e diminuindo o desempenho geral do sistema. Uma 
alternativa para esse problema seria a adoção da biblioteca PHPJasperXML 
(www.simitgroup.com/?q=PHPJasperXML), que gera relatórios usando o próprio 
interpretador do PHP a partir da leitura dos arquivos XML gerados e utilizados 
pelo Jaspereport na criação de relatórios.\\
A utilização da \textit{Hip-Hop Virtual Machine} - HHVM (www.hhvm.com), máquina 
virtual desenvolvida pelo Facebook para ser utilizada nos servidores da empresa 
e de código aberto, possibilita mais uma oportunidade de melhoria no desempenho 
de sistemas desenvolvidos em PHP, pois a ferramenta provê uma compilação em 
tempo real (\textit{just-in-time}) preservando a dinamismo do desenvolvimento 
em PHP.