\chapter{Conclusão}\label{cap:conclusao}
Com base nos dados coletados e analisados, é possível dizer que o servidor HTTP 
Nginx conseguiu ser, em média, 5 vezes mais eficiente que o servidor HTTP 
Apache, dando uma margem de sobrevida ao SIGA da forma como o sistema é 
desenvolvido hoje, usando o \textit{framework} Miolo. Porém, mesmo com a 
utilização de um servidor HTTP mais eficiente, é difícil dizer até quando o 
SIGA conseguirá escalar em relação a quantidade de usuários ativos.\\
Como dito na introdução, escalabilidade é um atributo desejável de um sistema. 
Ao meu ver, o \textit{framework} Miolo não oferece um nível de escalabilidade 
suficientemente bom, sendo necessário investimento em novos servidores com 
desempenho superior ao que é utilizado atualmente, para que o sistema suporte 
uma grande demanda. O Miolo, ao ``esconder'' o código HTML do desenvolvedor com 
a intenção de facilitar e agilizar o desenvolvimento, acaba exigindo uma alta 
carga de processamento devido ao seu mecanismo de geração de páginas 
dinâmicas.\\
Além da substituição do servidor HTTP para o Nginx, o ideal seria a total 
substituição do SIGA, sendo desenvolvido do zero, podendo utilizar a mesma 
linguagem de programação (PHP) porém com um \textit{framework} mais atual e em 
constante desenvolvimento, reduzindo a curva de aprendizado. Alguns dos  
\textit{frameworks} PHP mais utilizados hoje são:
\begin{itemize}
	\item Zend Framework
	\item CakePHP
	\item Code Igniter
	\item Laravel
\end{itemize}
Uma substituição mais radical envolveria a mudança até mesmo da linguagem de 
programação utilizada no desenvolvimento, sendo Ruby junto com o 
\textit{framework} Ruby on Rails, e a linguagem Python junto com o 
\textit{framework} Django, os mais recomendados, por serem utilizadas com 
sucesso por várias empresas, serem de código aberto (sem custo com licenças) e 
contam com uma grande comunidade de desenvolvedores.
\section{Trabalhos futuros}
A troca do servidor HTTP, do Apache para o Nginx, não é a única coisa que pode 
ser feita para que o SIGA não tenha o seu ciclo de vida encerrado em um futuro 
próximo. A atualização dos \textit{softwares} utilizados pelo SIGA traria não 
somente melhoria no desempenho, como também mais segurança. A versão do PHP 
utilizada no servidor do SIGA a 5.3.3, lançada em Junho de 2.009, sendo versão 
estável atual 5.6, lançada em Outubro de 2.014. A versão do 
\textit{framework} Miolo utilizado no SIGA é a 2.0, sendo a versão estável 
atual 2.6.\\
A ferramenta de relatórios utilizada representa outro gargalo importante no 
desempenho do SIGA. O Jaspereport foi desenvolvido originalmente para ser 
utilizado com a linguagem Java, sendo que, para cada relatório gerado, uma 
máquina virtual Java é criada para cada relatório gerado, consumindo os 
recursos do servidor e degradando o desempenho geral do sistema. Uma 
alternativa para esse problema seria a adoção da biblioteca PHPJasperXML 
(www.simitgroup.com/?q=PHPJasperXML), que gera relatórios usando o próprio 
interpretador do PHP a partir da leitura dos arquivos XML utilizados 
pelo Jaspereport na criação de relatórios.\\
A utilização da \textit{Hip-Hop Virtual Machine} - HHVM (www.hhvm.com), máquina 
virtual desenvolvida pelo Facebook para ser utilizada nos servidores da empresa 
e de código aberto, possibilita mais uma oportunidade de melhoria no desempenho 
de sistemas desenvolvidos em PHP, pois a ferramenta provê uma compilação em 
tempo real (\textit{just-in-time}), preservando a dinamismo do desenvolvimento 
em PHP.
