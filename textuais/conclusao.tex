\chapter{Conclusão}\label{cap:conclusao}
Ao analisar os dados demonstrados no capítulo \ref{cap:analise-dos-dados}, fica 
claro quanto o Nginx consegue ser mais eficiente do que o Apache. Em todas as 
métricas analisadas, o Nginx teve um desempenho superior ao obtido nos testes 
com o Apache.\\
Levando em consideração que os testes foram realizados em máquinas virtuais com 
desempenho limitado, é de se imaginar que, quando utilizado em um computador 
com desempenho superior, como os computadores servidores utilizados na 
universidade, o Nginx tera um desempenho semelhante ou até mesmo superior ao 
que foi aferido neste estudo.
\section{Novas melhorias}
A troca do servidor HTTP, do Apache para o Nginx, não é a única coisa que pode 
ser feita para que o SIGA não tenha o seu ciclo de vida encerrado em um futuro 
próximo.\\
A atualização dos \textit{softwares} utilizados pelo SIGA traria não somente 
uma melhoria no desempenho, como também em segurança. A versão do PHP utilizada 
no servidor do SIGA a 5.3.3, lançada em Junho de 2.009, 5 anos atrás, sendo 
compatível com versão 5.5.9 do PHP que está presente no repositório oficial da 
distribuição Linux Ubuntu 14.04 LTS. A versão do \textit{framework} Miolo 
utilizado no SIGA é a 2.0, sendo que a versão mais atual disponível é a 2.6.\\


O Nodejs é um servidor HTTP baseado no interpretador da linguagem JavaScript 
V8, utilizada hoje no navegador Google Chrome e no Chromium
nodejs como servidor http\\
A substituição da tecnologia utlizada para geração de relatórios utilizada no 
SIGA deve, 
jaspereport - classe php para analisar jrxml\\
phpjasperxml
hhvm

a morte do SIGA\\
