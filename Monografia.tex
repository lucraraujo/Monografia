\documentclass[
	% -- opções da classe memoir --
	12pt,				% tamanho da fonte
	%openright,			% capítulos começam em pág ímpar (insere página vazia caso preciso)
	oneside,			% para impressão em verso e anverso. Oposto a oneside
	a4paper,			% tamanho do papel. 
	% -- opções da classe abntex2 --
	%chapter=TITLE,		% títulos de capítulos convertidos em letras maiúsculas
	%section=TITLE,		% títulos de seções convertidos em letras maiúsculas
	%subsection=TITLE,	% títulos de subseções convertidos em letras maiúsculas
	%subsubsection=TITLE,% títulos de subsubseções convertidos em letras maiúsculas
	% -- opções do pacote babel --
	english,			% idioma adicional para hifenização
	brazil				% o último idioma é o principal do documento
]{abntex2}

% ---
% Pacotes básicos 
% ---
\usepackage{lmodern}				% Usa a fonte Latin Modern			
\usepackage[T1]{fontenc}			% Selecao de codigos de fonte.
\usepackage[utf8]{inputenc}		% Codificacao do documento (conversão automática dos acentos)
\usepackage{lastpage}			% Usado pela Ficha catalográfica
\usepackage{indentfirst}			% Indenta o primeiro parágrafo de cada seção.
\usepackage{color}				% Controle das cores
\usepackage{graphicx}			% Inclusão de gráficos
\usepackage{microtype}			% para melhorias de justificação
\usepackage{multirow}			% mesclar linhas em tabelas
\usepackage{comment}				% habilita comentário em bloco
\usepackage{float}

% ---
% Pacotes de citações
% ---
%\usepackage[brazilian,hyperpageref]{backref}		% Paginas com as citações na bibl
\usepackage[alf]{abntex2cite}					% Citações padrão ABNT

% --- 
% CONFIGURAÇÕES DE PACOTES
% --- 
% Configurações do pacote backref
% Usado sem a opção hyperpageref de backref
%\renewcommand{\backrefpagesname}{Citado na(s) página(s):~}
% Texto padrão antes do número das páginas
%\renewcommand{\backref}{}
% Define os textos da citação
%\renewcommand*{\backrefalt}[4]{
%	\ifcase #1 %
%		Nenhuma citação no texto.%
%	\or
%		Citado na página #2.%
%	\else
%		Citado #1 vezes nas páginas #2.%
%	\fi}%

% ---
% Informações de dados para CAPA e FOLHA DE ROSTO
% ---
\titulo{Proposta de melhoria de desempenho do SIGA utilizando Nginx como servidor HTTP}
\autor{Lucas Rafael Araujo Andrade}
\local{Diamantina}
\data{Dezembro de 2014}
\orientador{Professor Dr. Alexandre Ramos Fonseca}
\instituicao{%
  Universidade Federal dos Vales do Jequitinhonha e Mucuri - UFVJM
  \par
  Faculdade de Ciências Exatas e Tecnológicas - FACET
  \par
  Departamento de Computação - DECOM
  \par
  Bacharelado em Sistemas de Informação}
\tipotrabalho{Trabalho de Conclusão de Curso de Gradua}
% O preambulo deve conter o tipo do trabalho, o objetivo, 
% o nome da instituição e a área de concentração 
\preambulo{Trabalho de conclusão de curso apresentado à banca examinadora como parte dos requisitos da disciplina de Projeto Orientado II para obtenção do grau de Bacharel em Sistemas de Informação.}
% ---


% ---
% Configurações de aparência do PDF final

% informações do PDF
\makeatletter
\hypersetup{
     	%pagebackref=true,
		pdftitle={\@title}, 
		pdfauthor={\@author},
    		pdfsubject={\imprimirpreambulo},
	    pdfcreator={LaTeX with abnTeX2},
		pdfkeywords={NGINX}{APACHE2}{SIGA}{UFVJM}{Monografia}, 
		colorlinks=false,       		% false: boxed links; true: colored links
    		%linkcolor=blue,          	% color of internal links
    		%citecolor=blue,        		% color of links to bibliography
    		filecolor=magenta,      		% color of file links
		%urlcolor=blue,
		bookmarksdepth=4,
		pdfborder = {0 0 0}
}
\makeatother

% --- 
% Espaçamentos entre linhas e parágrafos 
% --- 
% O tamanho do parágrafo é dado por:
\setlength{\parindent}{1.3cm}
% Controle do espaçamento entre um parágrafo e outro:
\setlength{\parskip}{0.2cm}  % tente também \onelineskip

% ---
% compila o indice
% ---
\makeindex

% ----
% Início do documento
% ----
\begin{document}

% Retira espaço extra obsoleto entre as frases.
\frenchspacing 

% ----------------------------------------------------------
% ELEMENTOS PRÉ-TEXTUAIS
% ----------------------------------------------------------
\pretextual

% ---
% Capa
% ---
\imprimircapa

% ---
% Folha de rosto
% ---
\imprimirfolhaderosto

% ---
% Inserir folha de aprovação
% ---
\begin{folhadeaprovacao}

  \begin{center}
    {\ABNTEXchapterfont\large\imprimirautor}

    \vspace*{\fill}\vspace*{\fill}
    \begin{center}
      \ABNTEXchapterfont\bfseries\Large\imprimirtitulo
    \end{center}
    \vspace*{\fill}
    
    \hspace{.45\textwidth}
    \begin{minipage}{.5\textwidth}
        \imprimirpreambulo
    \end{minipage}%
    \vspace*{\fill}
   \end{center}
        
   APROVADO em:        /        /

 
   \assinatura{\textbf{Professor} \\ Universidade Federal dos Vales do Jequitinhonha e Mucuri}
   \assinatura{\textbf{Professor} \\ Universidade Federal dos Vales do Jequitinhonha e Mucuri}
   \assinatura{\textbf{\imprimirorientador} \\ Orientador}
      
   \begin{center}
    \vspace*{0.5cm}
    {\large\imprimirlocal}
    \par
    {\large\imprimirdata}
    \vspace*{1cm}
  \end{center}
  
\end{folhadeaprovacao}

% ---
% Dedicatória
% ---
\begin{dedicatoria}
   \vspace*{\fill}
   \centering
   \noindent
   \textit{Dedico esse trabalho à mulher que tornou isso tudo possível. Te Amo Mãe!} \vspace*{\fill}
\end{dedicatoria}

% ---
% Agradecimentos
% ---
\begin{agradecimentos}

\begin{comment}

Agradeço primeiramente aos meus mestres que em todos esses anos se emprenharam em passar o máximo de conhecimento possível, em especial o meu orientador Alexandre pela oportunidade de desenvolver esse projeto.\\
\\
Agradeço aos meus amigos de Bocaiuva e Diamantina, Arthur, Leandro, Thalita, Diego, Hebert, Mayra, Natália, Fernanda, Lorena e Paulo Barreto. Saibam que todos, de alguma forma, fizeram diferença na minha vida e com todos vocês eu aprendi algo.\\
\\
Agradeço as minhas amigas de Winnipeg: Tamara, Carol, Stefanie, Juliana e Ingrid. Com vocês por perto tudo se tornou menos difícil e mais alegre durante a minha estadia no Canadá.\\

Agradeço a Lili pelo apoio durante os mais de dois anos em que estivemos juntos. Saiba você que teve um papel muito importante na minha vida e lembrarei sempre dos nossos momentos juntos.\\
\\
Agradeço aos meus familiares que me diretamente ou indiretamente me ajudaram durante esse tempo todo em Diamantina, Tio Vendoval, Tia Letícia (In memorian), Vó Nita e Tio Antônio.\\

Agradeço aos Técnicos-Administrativos do DTI da UFVJM, Everton, William, Marcelo, Rodrigo , André e Ricardo Brasil, por todo o conhecimento e experiência compartilhados com nós, estagiários.\\

E por último, porém mais importante, a mulher que por todos os meus 26 anos de vida me aturou com um amor incondicional que somente uma Mãe é capaz de sentir. Saiba, Mãe, que sem você, nada disso seria possível. Te amo incondicionalmente também!\\

\end{comment}
\end{agradecimentos}

% ---
% Epígrafe
% ---
\begin{epigrafe}
    \vspace*{\fill}
	\begin{flushright}
		\textit{``Code is Poetry''\\
		(Wordpress.org)}
	\end{flushright}
\end{epigrafe}

% ---
% Resummo
% ---
\setlength{\absparsep}{18pt} % ajusta o espaçamento dos parágrafos do resumo
\begin{resumo}
 \textbf{Palavras-chaves}: Apache. Nginx. SIGA. UFVJM.
\end{resumo}

% ---
% Abstract
% ---
\begin{resumo}[Abstract]
 \begin{otherlanguage*}{english}
The goal of this work is to identify the Nginx HTTP server efficiency, for 
processing requisitions compared to the Apache HTTP server, currently in use by 
the Federal University of the Vales of Jequitinhonha and Mucuri – UFVJM in its 
Academic Integrated Management System - SIGA. Both server studied are among the 
seven most used in the world and among the five most used by the most accessed 
web sites wide world. The data has been collected after performance tests 
performed with the ApacheBench tool, installed in two virtual machines, equally 
configured and installed in the visualization software VirtualBox hosted in a 
portable computer.\\
Based on the amount of users in the academic community oh the UFVJM, was 
defined fifteen tests values, four simultaneous access scenarios and six 
efficiency metrics. In order to provide a better visualization of the results 
acquired, the collected data was distributed in tree graphs divided in tree 
ranges of values an a graph with the total amount of requisitions tested. The 
result analysis was made for each metric in an individual way, showing the 
average performance from each server. After the analysis, the conclusion about 
the Nginx and Apache HTTP server is shown and concludes about which server is 
more adequate e efficient to SIGA.
   \vspace{\onelineskip}
 
   \noindent 
   \textbf{Key-words}: Apache. Nginx. SIGA. UFVJM.
 \end{otherlanguage*}
\end{resumo}

% ---
% Lista de Figuras
% ---
\pdfbookmark[0]{\listfigurename}{lof}
\listoffigures*
\cleardoublepage

% ---
% Lista de Tabelas
% ---
\pdfbookmark[0]{\listtablename}{lot}
\listoftables*
\cleardoublepage

% ---
% Siglas e Abreviaturas
% ---
\begin{siglas}
  \item[API] \textit{Application Programming Interface}
  \item[CGI] \textit{Common Gateway Interface}
  \item[HTML] \textit{Hypertext Markup Language}
  \item[HTTP] \textit{Hipertext Transport Protocol}
  \item[ODBC] \textit{Open Database Connection}
  \item[PDI] Plano de Desenvolvimento Institucional
  \item[PDO] \textit{PHP Data Object}
  \item[PHP] \textit{Hypertext Preprocessor}
  \item[PHP-FPM] \textit{PHP FastCGI Process Manager}
  \item[SIGA] Sistema Integrado de Gestão Acadêmica
  \item[SGBD] Sistema de Gerenciador de Banco de Dados
  \item[TCP] \textit{Transmission Control Protocol}
  \item[TCP/IP] \textit{Transmission Control Protocol/Internet Protocol	}
  \item[URI] \textit{Uniform Resource Identifier}
  \item[URL] \textit{Uniform Resource Locator}
  \item[UFVJM] Universidade Federal dos Vales do Jequitinhonha e Mucuri
  \item[WWW] \textit{World Wide Web}
  \item[XML] \textit{Extensible Markup Language}
  \item[XHTML] \textit{Extensible Hypertext Markup Language}
\end{siglas}

% ---
% Sumário
% ---
\pdfbookmark[0]{\contentsname}{toc}
\tableofcontents*
\cleardoublepage

% ----------------------------------------------------------
% ELEMENTOS TEXTUAIS
% ----------------------------------------------------------
\textual

% ---
% Introducao
% ---
\chapter{Introdução}\label{introducao}
Com o avanço dos computadores e das redes de comunicação, os sistemas de informação foram sendo transferidos dos computadores pessoais para os servidores \textit{web}, de onde eles podem ser acessados, virtualmente, a partir de todo o mundo. Para que esses sistemas funcionem, eles precisam de \textit{softwares} que atendam as requisições que chegam ao servidor a partir da rede de computadores usando o protocolo HTTP. Esses softwares são chamados Servidores HTTP.\\
Hoje o SIGA – Sistema Integrado de Gestão Acadêmica, desenvolvido utilizando a linguagem de programação PHP e o framework Miolo, utiliza o Apache HTTP \textit{Server} como servidor HTTP.\\

\section{Objetivos}
O objetivo desse estudo é identificar se a utilização do servidor HTTP Nginx é mais eficiente do que o que é utilizado atualmente, o Apache HTTP \textit{Server}.\\

\section{Motivação}
Com a expansão das Universidades Federais, por ano são admitidos 3500 alunos na graduação, pós-graduação e graduação e especialização à distância, além de novos servidores técnicos-administrativos e professores. No final de 2.013, data do último levantamento, a Universidade Federal dos Vales do Jequitinhonha e Mucuri tinha 8.121 alunos, 576 professores e 421 servidores técnicos-administrativos, espalhados em quatro \textit{campi} universitários, fazendas experimentais e pólos de ensino em educação à distância, totalizando 9.118 pessoas que interagem com a universidade diariamente.\\
Em épocas de pico de utilização do SIGA, as reclamações de lentidão e problemas no sistemas são frequentes, as vezes impossibilitando a utilização do mesmo. Os picos mais notório são: fim de período letivo da graduação, quando alunos e professores acessam o sistema para olhar e lançar notas, respectivamente; e rematricula dos alunos da graduação, quando os mesmos escolhem as matérias que desejam cursar no período seguinte.\\
Com o crescente aumento de alunos, servidores públicos (professores e técnicos-administrativos) e teceirizados na universidade, a tendência é que a utilização do SIGA se torne mais problemática.
Com isso em mente, a utilização do servidor HTTP Nginx pode ajudar a amenizar os problema de desempenho do SIGA.\\


% ---
% Fundamentacoes Teoricas
% ---
\chapter{Fundamentação Teórica}\label{cap:fundamentacao-teorica}
Este capítulo se destina a apresentar alguns conceitos para o entendimento da 
proposta desse estudo.
\section{Aplicação cliente/servidor}
Para entender o que é uma aplicação cliente/servidor, irei primeiramente definir o que é cliente e servidor.
\subsection{Cliente}
Cliente é:
\begin{citacao}
Um solicitante de informações em rede, normalmente um PC ou estação de trabalho, que pode consultar o banco de dados e/ou outras informações de um servidor. \cite{stallings2005}
\end{citacao}
Ainda de acordo com \citeonline{stallings2005}, o cliente é um computador ou estação de trabalho, que apresenta ao usuário final uma interface gráfica de alto nível e amigável, incluindo o uso de janelas e \textit{mouse}. 
\subsection{Servidor}
Servidor é:
\begin{citacao}
Um computador, normalmente uma estação de trabalho poderosa ou um \textit{mainframe} que abriga informações para manipulação por clientes em rede. \cite{stallings2005}
\end{citacao}
Ainda de acordo com \citeonline{stallings2005}, cada servidor no ambiente cliente/servidor fornece um conjunto de serviços compartilhados.
\subsection{Aplicações cliente/servidor}
As aplicações cliente/servidor são um conjunto de programas que funcionam no computador cliente, ou seja, aquele que esta solicitando um recurso pela rede, e no computador servidor, aquele que provê o recurso solicitado. As solicitações de recursos ou informações partem do programa instalado no computador do usuário (cliente), trafegam por uma rede de computadores, geralmente utilizando o protocolo de comunicação HTTP até o servidor responsável por prover o o recurso ou informação solicitado pelo programa do cliente.\\
As aplicações cliente/servidor são muito utilizadas na construção de sistemas de informação, onde os dados armazenados pelos servidores precisam estar disponíveis para os usuários do sistema. É também o principal modelo utilizado em sistemas de informação baseados na \textit{web}, onde o cliente acessa o sistema através de um navegador \textit{web}.
\section{Protocolo de comunicação}
De acordo com o \citeonline{dop}, um dos significados de protocolo é Acordo 
regulamentado entre países ou empresas. Aplicando esse conceito na computação, 
protocolo pode ser entendido como uma convenção que controla e possibilita uma 
conexão, comunicação, transferência de dados entre dois sistemas 
computacionais. De maneira simples, um protocolo pode ser definido como as 
regras que governam a sintaxe, semântica e sincronização da comunicação.\\
De acordo com \citeonline{stallings2005}, um protocolo é usado para a 
comunicação entre entidades em sistemas diferentes. Para que entidades 
diferentes se cominiquem com facilidade, elas precisam adotar uma ¨língua¨ em 
comum. O que e como é comunicado deve estar de acordo com as convenções 
combinadas entre as entidades envolvidas. Essas convenções são denominadas 
protocolos, que nada mais é do que um conjunto de regras controlando a troca de 
dados entre entidades. Os principais elementos de um protocolo são:
\begin{itemize}
	\item Sintaxe: Inclui elementos como formato de dados e níveis de sinal;
	\item Semântica: Inclui informações de controle para coordenação e 
	tratamento de erro;
	\item Temporização: Inclui combinação de velocidade e sequência.
\end{itemize}
\section{Modelo TCP/IP}
O TCP/IP (\textit{Transmission Control Protocol/Internet Protocol}), é um conjunto de protocolos, resultante de pesquisas e desenvolvimentos realizados pela ARPANET (\textit{Advanced Research Project Agency Network}) e financiada pela DARPA (\textit{Defense Advanced Research Projects Agency}), agência de pesquisa militar do governo dos Estados Unidos, e que hoje são utilizados como padrões de comunicação na Internet.\\
Ao contrário do modelo OSI(Open Systems Interconnections), o TCP/IP não pode ser considerado um modelo e sim um conjunto de protocolos. Esse protocolos, assim como no modelo OSI, podem ser organizados camadas relativamente independentes, com a diferença de que no modelo OSI existem sete camadas e no TPC/IP existem cinco, camadas essas relativamente independentes. São elas:
\begin{itemize}
	\item Camada de aplicação
	\item Camada de transporte
	\item Camada de rede
	\item Camada de enlace
	\item Camada física
\end{itemize}
Só para efeito de comparação, as camadas do modelo OSI são:
\begin{itemize}
	\item Camada de aplicação
	\item Camada de apresentação
	\item Camada de sessão
	\item Camada de transporte
	\item Camada de rede
	\item Camada de enlace
	\item Camada física
\end{itemize}
%aplicação
\subsection{Camada de aplicação}
%host a host
\subsection{Camada de transporte}
%inter-rede
\subsection{Camada de rede}
%acesso à rede
\subsection{Camada de enlace}
A camada de enlace (ou de acesso à rede) trata da troca de dados entre um 
sistema final e a rede à qual está conectada \cite{stallings2005}. Para que 
haja uma comunicação, o computador de origem da mensagem deve informar para a 
camada de enlace qual o endereço do computador de destino, para que a rede 
possa entregar a mensagem para o destinatário correto. O computador de origem 
pode requerer alguns serviços especiais, como por exemplo, prioridade no envio 
da mensagem. O software utilizado nessa camada irá depender do tipo de rede a 
ser usado, sendo que diferentes padrões foram desenvolvidos para comutação de 
circuitos pacotes.\\
Quando dois computadores estão conectado a redes diferentes, se faz necessário 
o uso de procedimentos para que os dados trafeguem entre as redes. Nesses 
casos, se utiliza o protocolo IP (\textit{Internet Protocol}). Esse protocolo 
oferece a função de interconectar várias redes através de roteadores. Os 
roteadores são computadores especializados em conectar duas ou mai redes 
diferentes, e o protocolo IP é também utilizado e implementado nos roteadores.
\subsection{Camada física}


\section{Protocolo TCP}
\section{\textit{World Wide Web}}
\section{Protocolo HTTP}
De acordo com \citeonline{stallings2005}, o \textit{Hypertext Transfer Protocol} (Protocolo de Transferência de Hipertexto em tradução livre) é o protocolo básico da \textit{World Wide Web} e pode ser usado em qualquer aplicação cliente/servidor que envolve hipertexto e faz parte da camada de transporte do modelo TCP/IP.\\
O HTTP não serve somente para transferir hipertexto. É um protocolo para transferir informações com uma eficiência necessária. Os dados transferidos podem ser texto puro, hipertexto, áudio, imagens ou qualquer outro dado disponível na Internet.\\
O HTTP é um protocolo cliente/servidor orientado para transações. O seu uso mais típico é acontece entre um navegador \textit{web} e um servidor \textit{web}, sempre utilizando o modelo TCP afim de manter a confiabilidade.\\
Cada transação utiliza uma nova conexão TCP entre o cliente e o servidor, sendo cada transação tratada de forma independente. Após o término de cada transação, a conexão entre cliente e servidor é fechada. Por ser um protocolo sem estado, é o protocolo adequado para a maior parte das aplicações.
\section{\textit{Framework}}
\section{Servidor \textit{web}}
\section{C10K \textit{problem}}
O C10k \textit{problem} consiste em solucionar o problema de atender a 10.000 requisições concorrentemente.\\
\textbf{Escrever sobre}

% ---
% Tecnologias Utilizadas
% ---
\chapter{Tecnologias Utilizadas}\label{cap:tecnologias_utilizadas}
\section{Apache HTTP \textit{Server}}
O Apache HTTP \textit{Server} teve o seu primeiro lançamento publico em Abril de 1.995. Ele foi criado para ocupar o lugar deixado pelo HTTP \textit{Daemon}, na época o servidor para aplicações web mais utilizado no mundo. O HTTP \textit{Daemon} foi desenvolvido por Rob McCool quando ele trabalhava no \textit{National Center for Supercomputing Applications} – NCSA, na Universidade de Illinois, nos Estados Unidos. Porem, o desenvolvimento do HTTP \textit{Daemon} estagnou-se pois McCool havia saído da universidade. Como o código do HTTP \textit{Daemon} era aberto (\textit{open source}), vários desenvolvedores criaram correções e desenvolveram novas funcionalidades para o mesmo. Vendo a necessidade de juntar todos esses códigos desenvolvidos em separado, um grupo de desenvolvedores resolveram se juntar para compilar essas correções e novas funcionalidades. Usando como base a versão 1.3 do HTTP \textit{Daemon}, em Abril de 1.995 foi publicado o Apache HTTP \textit{Server} na versão 0.6.2. Também, nessa mesma época, foi criado o Apache \textit{Group}, grupo que mais tarde viria a se tornar o Apache \textit{Software Foundation}.\\
Hoje, quase 20 anos após o seu primeiro lançamento, o Apache HTTP \textit{Server} é o servidor HTTP mais utilizado no mundo e a sua versão estável atual é a 2.4.\\
\section{Nginx}
O Nginx (lê-se \textit{Engine-X}) foi criado pelo russo Igor Sysoev em 2.002 tendo a primeira versão publica sendo publicada em 2.004. O Nginx foi desenvolvido com o intuito de resolver o C10K \textit{problem}.\\
Diferentemente de outros servidores HTTP, o Nginx não usa \textit{threads} como base para manipular as requisições. Ao invés disso, ele utiliza uma arquitetura mais escalável orientada à eventos (\textit{event-driven}) assíncrona. Essa arquitetura utiliza uma quantidade pequena, porém previsível, de memória quando está trabalhando.\\
\begin{figure}[h!]
\centering
\includegraphics[scale=1]{figuras/nginx-how-it-works} 
\caption{Modelo de funcionamento do Nginx.}
\label{fig:nginx-comofunciona}
\end{figure}
O Nginx é utilizado por vários sítios de grande volume de tráfego como Netflix, GitHub, Pinterest, dentre outros.\\
\section{ApacheBench}
O ApacheBench foi criado em 1996 por Adam Twiss e, posteriormente doado ao Apache \textit{Group}. Originalmente, essa ferramenta foi desenvolvida para verificar o desempenho em servidores HTTP Apache, mas hoje ela é utilizada para fazer testes de desempenho em  praticamente qualquer servidor HTTP.
\section{FastCGI}
De acordo com \citeonline{fastcgi} FastCGI é uma interface para servidores \textit{web} rápida, aberta e segura, que resolve os problemas de desempenho herdados do CGI, sem introduzir o \textit{overhead} e complexidade de APIs proprietárias.
\subsection{\textit{Common Gateway Interface}}
A interface de fato de aplicações em servidores \textit{web} é o CGI, que foi primeiramente implementado no servidor da NCSA. O CGI tem muitos benefícios:
\begin{itemize}
	\item Simplicidade: é fácil de entender;
	\item Independente de linguagem: Aplicações em CGI podem ser escritas em quase todas a linguagens;
	\item Isolamento do processo: Com os processos são executados em processos separados, aplicações com problemas não podem para o servidor \textit{web} ou acessar o estado interno do servidor;
	\item Padrão aberto: Alguma forma de CGI já foi implementado em todos os servidores;
	\item Independência de arquitetura: O CGI não é ligado a uma arquitetura de computador em particular.
\end{itemize}
CGI tem alguns inconvenientes significantes. O principal problema é desempenho: como um novo processo é criado para cada requisição e descartada quando a requisição acaba,a eficiência é baixa.
\subsection{Servidor de API}
Em resposta ao problema de desempenho do CGI, várias empresas desenvolveram API's para os seus servidores.\\
Aplicações conectadas em um servidor de API pode ser significativamente mais rápido do que programas em CGI. O problema da inicialização do CGI é melhorada, pois a aplicação é executada no processo do servidor e persiste pelas requisições. As API's dos servidores \textit{web} também oferecem mais funcionalidades do que o CGI. O desenvolvedor pode criar extensões que permitem realizar controle de acesso, pegar um acesso dos aquivos de registro(\textit{log}) do servidor e, se conectar a outros estágios do processamento de uma requisição do servidor.\\
No entanto, API's sacrificam todos os benefícios do GCI. São eles:
\begin{itemize}
	\item Complexidade: API's de empresas introduzem uma curva de aprendizado, com curtos de implementação e manutenção maiores;
	\item Dependencia de linguagem: as aplicações devem ser escritas na linguagem suportada pelo desenvolvedor da API;
	\item Não há isolamento de processos: como o processo é executado dentro do endereçamento de memória do servidor, aplicações com problemas podem corromper o núcleo do servidor, comprometer a segurança e problemas no núcleo do servidor podem corromper as aplicações;
	\item Proprietário: Codificar a aplicação para uma determinada API força o desenvolvedor a utilizar aquele servidor em particular;
	Arquiteturas de computador iguais: As aplicações tem que compartilhar sa mesma arquitetura do servidor \textit{web}.
\end{itemize}
\subsection{FastCGI}
A interface do FastCGI combina os melhores aspectos do CGI e das API's proprietárias. Assim como o CGI, o FastCGI executa os processos de forma separada e isolada. As vantagens do FastCGI incluem:
\begin{itemize}
	\item Desempenho: Os processos do FastCGI são persistentes. Eles são reutilizados para manipular várias requisições. Isso resolve o problema de criar um novo processo para cada requisição;
	\item Simplicidade com fácil migração do CGI: a biblioteca de aplicação do FastCGI simplifica a migração de aplicações existentes feitas usando CGI. Aplicações feitas utilizando a biblioteca de aplicações do FastCGI podem ser executadas como programa CGI;
	\item Independente de linguagem: Assim como o CGI, as aplicações FastCGI podem ser escritas em qualquer linguagem;
	\item Isolamento de processos: Uma aplicação com problemas não pode corromper o núcleo do servidor ou de outra aplicação. Uma aplicação maliciosa não pode roubar informações do servidor \textit{web};
	\item Independência de arquitetura: O FastCGI não é ligado a uma arquitetura de computador em particular. Qualquer servidor \textit{web} pode implementar a interface do FastCGI;
	\item Suporte à computação distribuída: FastCGI provê a habilidade de executar aplicações remotamente, o que é útil para distribuir carga e gerenciar sítios da \textit{web} externos
\end{itemize}
\section{HTML}
De acordo com o \citeonline{w3chtml} a Web é baseada em 3 pilares:
\begin{itemize}
\item Um esquema de nomes para localização de fontes de informação na Web, esse esquema chama-se URI.
\item Um Protocolo de acesso para acessar estas fontes, hoje o HTTP.
\item Uma linguagem de Hipertexto, para a fácil navegação entre as fontes de informação: o HTML.
\end{itemize}
\subsection{Hipertexto}
HTML é uma abreviação de \textit{Hypertext Markup Language} - Linguagem de Marcação de Hipertexto. Resumindo em uma frase: o HTML é uma linguagem para publicação de conteúdo (texto, imagem, vídeo, áudio e etc) na \textit{web}.\\
O HTML é baseado no conceito de Hipertexto. Hipertexto são conjuntos de elementos – ou nós – ligados por conexões. Estes elementos podem ser palavras, imagens, vídeos, áudio, documentos etc. Estes elementos conectados formam uma grande rede de informação. Eles não estão conectados linearmente como se fossem textos de um livro, onde um assunto é ligado ao outro seguidamente. A conexão feita em um hipertexto é algo imprevisto que permite a comunicação de dados, organizando conhecimentos e guardando informações relacionadas.\\
Para distribuir informação de uma maneira global, é necessário haver uma linguagem que seja entendida universalmente por diversos meios de acesso. O HTML se propõe a ser esta linguagem. 
Desenvolvido originalmente por Tim Berners-Lee o HTML ganhou popularidade quando o \textit{Mosaic - browser} desenvolvido por Marc Andreessen na década de 1990 - ganhou força. A partir daí, desenvolvedores e fabricantes de navegadores utilizaram o HTML como base, compartilhando as mesmas convenções.\\
\subsection{HTML}
Entre 1993 e 1995, o HTML ganhou as versões HTML+, HTML2.0 e HTML3.0, onde foram propostas diversas mudanças para enriquecer as possibilidades da linguagem. Contudo, até aqui o HTML ainda não era tratado como um padrão. Apenas em 1997, o grupo de trabalho do W3C responsável por manter o padrão do código, trabalhou na versão 3.2 da linguagem, fazendo com que ela fosse tratada como prática comum.\\
Desde o começo o HTML foi criado para ser uma linguagem independente de plataformas, navegadores e outros meios de acesso. Interoperabilidade significa menos custo. Você cria apenas um código HTML e este código pode ser lido por diversos meios, ao invés de versões diferentes para diversos dispositivos. Dessa forma, evitou-se que a Web fosse desenvolvida em uma base proprietária, com formatos incompatíveis e limitada.\\
Por isso o HTML foi desenvolvido para que essa barreira fosse ultrapassada, fazendo com que a informação publicada por meio deste código fosse acessível por dispositivos e outros meios com características diferentes, não importando o tamanho da tela, resolução, variação de cor. Dispositivos próprios para deficientes visuais e auditivos ou dispositivos móveis e portáteis. O HTML deve ser entendido universalmente, dando a possibilidade para a reutilização dessa informação de acordo com as limitações de cada meio de acesso.
\section{PHP}
PHP (acrônimo para preprocessador de hipertexto) é uma linguagem de programação de código aberto, interpretada, de propósito geral, tipagem dinâmica e fraca, procedural, reflexiva, orientada a objetos e funcional; criada em 1.995 por Rasmus Lerdorf. É uma linguagem que é melhor utilizada para desenvolvimento de sistemas \textit{web} já que pode ser inserida diretamente em códigos HTML.\\
De acordo com \citeonline{phpwhatcando}, o PHP pode realizar qualquer tipo de atividade computacional. É focada em executar tarefa do lado do servidor, realizando qualquer tarefa que uma aplicação feita em CGI pode fazer tais como: coletar dados de um formulário, gerar páginas com conteúdo dinâmico ou enviar e recebe \textit{cookies}. Existem três áreas onde o PHP é mais utilizado:
\begin{itemize}
	\item \textit{Script} do lado do servidor: é a forma mais tradicional e o principal foco do PHP.
	\item \textit{Script} de linha de comando: é uma forma de utilizar o PHP sem um servidor \textit{web} ou navegador de internet. Esse forma de uso é ideal para rotinas programadas executadas no sistema operacional. Pode, ser usadas para, também, para tarefas de processamento de textos.
	\item Aplicações para \textit{Desktop}: PHP provavelmente não é a melhor linguagem para desenvolver aplicações com interface gráfica para computadores de mesa, mais pode ser utilizada para criação de aplicações do lado do cliente utilizando o a extensão PHP-GTK, que não é distribuída junto com a versão oficial da linguagem
\end{itemize}
É necessário três coisas para fazer o PHP funcionar. Um interpretador, um servidor \textit{web} e um navegador de \textit{internet}. ele pode ser utilizado em praticamente todos os sistemas operacionais, e pode ser executado em qualquer servidor \textit{web} que utilize o binário do FastCGI PHP.\\
Com o PHP, o desenvolvedor não fica limitado a somente gerar páginas HTML. As habilidades incluem a entrega de imagens e arquivo en geral, geração de páginas em XHTML e qualquer outro arquivo XML, podendo gerar esses arquivos de forma automática e salvando eles no dispositivo de armazenamento do servidor, servindo como \textit{cache} no servidor para o conteúdo gerado dinamicamente.\\
Uma das funcionalidades mais importantes do PHP é o suporte a uma grande variedade de bancos de dados. Desenvolver uma página que utiliza uma base de dados é simples, desde que se use uma das extensões para base de dados presente na linguagem ou se use uma camada de abstração, como por exemplo o PDO, ou conectar a qualquer base de dados que suporte o padrão ODBC via extensão.\\
O PHP ferramentas muito úteis de processamento de texto, que inclui um analisador de expressões regulares compatível com a linguagem de programação Perl além de várias extensões e ferramentas para analisar e acessar documentos no formato XML.
\subsection{PHP-FPM}
O PHP-FPM (\textit{(FastCGI Process Manager}) é uma implementação alternativa do PHP FastCGI com algumas funcionalidades úteis, principalmente, para sítios com grande acesso. Essas funcionalidade incluem:
\begin{itemize}
	\item Gerenciamento avançado de processos
	\item Habilidade para iniciar processos com diferentes usuários, grupos e ambientes, escutando portas diferentes e usando diferentes arquivos de configuração;
	\item Registro (\textit{log}) de atividades nas saídas padrão de texto (stdout) e de erros (stderr) dos sistema operacional;
	\item Reinicialização emergencial em caso de destruição acidental da memória cache;
\end{itemize}
\section{PostgreSQL}
De acordo com \citeonline{postgresql} PostgreSQL é um sistema gerenciador de banco de dados objeto-relacional que tem sido desenvolvido de várias formas desde 1.977. Começou como um projeto chamado Ingres na \textit{University of California} em Berkeley, Estados Unidos. O Ingres foi, posteriormente, desenvolvido comercialmente pela empresa Relational Technologies.\\
Em 1.986 uma outra equipe chefiada por Michael Stonebraker continuou o desenvolvimento do código do Ingres para criar um sistema de banco de dados usando o paradigma objeto-relacional chamado Postgres. Em 1.996, o Postgres foi renomeado para PostgreSQL.\\
O PostgreSQL é considerado por muitos o melhor SGBD de código aberto do mundo. Provê várias funcionalidades que, normalmente, são vistas somente em produtos comerciais desenvolvidos para corporações.\\
PostgreSQL é um projeto de código aberto. Por definição, código abertos ignifica que qualquer pessoa pode obter o código fonte, usar o programa e modificá-lo livremente sem se preocupar em infringir direitos autorais.
\section{Miolo \textit{framework}}

% ---
% Metodologia
% ---
\chapter{Metodologia}\label{cap:metodologia}
A descrição da metodologia utilizada para a coleta e análise dos dados é de 
extrema importância para o estudo. A partir dos dados coletado é que será 
possível determinar se os objetivos foram alcançados. Nesse capítulo irei falar 
sobre a metodologia de coleta e análise de dados.\\

\section{Coleta dos dados}
A coleta dos dados foram realizadas com o uso da ferramenta ApacheBench. O 
ApacheBench é uma ferramenta de para realizar testes de carga em servidores 
\textit{web} utilizada em linha de comando. Para invocar a ferramenta 
escreve-se o comando ``ab'' no terminal aceitando como parâmetros, entre 
outros, a quantidade total de requisições que serão feitas, quantas requisições 
serão feitas de forma simultânea e a URL da página que será requisitada. O 
comando de uma forma genérica fica assim:
\begin{verbatim}
ab -n numero_de_requisições -c requisições_simultâneas endereço_do_servidor
\end{verbatim}
O parâmetro ``-n'' indica a quantidade total de requisições serão feitas no 
teste e o parâmetro ``-c'' indica a quantidade de requisições serão feitas de 
forma simultânea. Ao fim da execução do teste, o ApacheBench retorna os os 
dados coletados.

\section{Ambiente de testes}

Para fazer os teste da forma mais neutra possível, foram criadas duas máquinas virtuais usando o software de virtualização VirtualBox na sua versão 3.4.10.\\
Todos os programas utilizados nos ambientes de teste foram instalados pelo gerenciador de pacotes nativo da distribuição Linux Debian \textit{aptitude} e disponíveis nos repositórios oficiais do Debian, exceto o SGBD PostgreSQL que foi instalado usando o \textit{aptitude} porém através do repositório oficial do PostgreSQL para o Linux Debian 7 pois nos repositórios do Debian não havia disponível a versão mais recente e utilizada pelo banco de dados do SIGA.

\subsection{Computador hospedeiro}
O computador onde foram instaladas as máquinas virtuais possui a seguinte 
configuração:
\begin{itemize}
	\item \textbf{Tipo}: Computador portátil (\textit{Notebook})
	\item \textbf{Processador}: Intel Core i5-3337U
	\item \textbf{Frequência do Processador}: 1,8 Giga Hertz
	\item \textbf{Tamanho da Memória Principal (RAM)}: 8 Giga Bytes;
	\item \textbf{Tamanho da Memória Secundária (HD)}: 1 Tera Bytes;
	\item \textbf{Quantidade de núcleos disponível para processamento}: 4 
	núcleos.
\end{itemize}

\subsection{Configurações comuns às duas máquinas virtuais}
 Ambas máquinas virtuais tinhas as configurações rigorosamente iguais. São elas:
\begin{itemize}
\item \textbf{Sistema Operacional}: Linux Debian 7 “Wheezy” 64 bits mais atualizado;
\item \textbf{Sistema Gerenciador de Banco de Dados}: PostgreSQL na versão 9.3.5
\item \textbf{Tamanho da Memória Principal (RAM)}: 2.048 Mega Bytes;
\item \textbf{Tamanho da Memória Secundária (HD)}: 30 Giga Bytes;
\item \textbf{Quantidade de núcleos disponível para processamento}: 1 núcleo.
\end{itemize}

\subsection{Máquina virtual Apache}
Na máquina virtual destinada aos testes com o servidor HTTP Apache, os programas utilizados foram:
\begin{itemize}
\item Apache HTTP Server na versão 2.2.22;
\item libapache2-mod-php5 na versão 5.4.4;
\item php5-common na versão 5.4.4.
\end{itemize}

\subsection{Máquina virtual Nginx}
Na maquina virtual destinada aos testes com o servidor HTTP Nginx, os programas utilizados foram:

\begin{itemize}
\item Servidor HTTP Nginx na versão 1.2.1;
\item php5-fpm na versão 5.4.4;
\item php5-common na versão 5.4.4.
\end{itemize}



\section{Definição dos valores utilizados nos testes}
De acordo com o último levantamento, a universidade conta com 7.252 
utilizadores, entre alunos, professores e funcionário técnicos-administrativos.
Com base nesse dado, foram determinados os valores utilizados nos testes, ou 
seja, cerca de 7.000 usuário ativos atualmente projetando mais que o dobro de 
usuário no futuro (15.000) e com cargas de acesso simultâneo de 10\%, 20\%, 
40\% e 80\% para cada quantidade total de utilizadores. Os valores utilizados 
nos testes estão descritos na tabela \ref{tab:requisicoes}.
\begin{table}[htb]
	\centering
\ABNTEXfontereduzida
\caption[Requisições Totais e Requisições Concorrentes]{Requisições Totais e Requisições Concorrentes}
\label{tab:requisicoes}
\begin{tabular}{|>{\bfseries}c|c|c|c|c|}
\hline
\multirow{2}{*}{Requisições Totais} & \multicolumn{4}{c|}{\textbf{Requisições 
Concorrentes}} \\ \cline{2-5}
& 10\%      & 20\%  & 40\%  & 80\%  \\ \hline
1.000  & 100   & 200   & 400   & 800   \\ \hline
2.000  & 200   & 400   & 800   & 1.600  \\ \hline
3.000  & 300   & 600   & 1.200 & 2.400  \\ \hline
4.000  & 400   & 800   & 1.600 & 3.200  \\ \hline
5.000  & 500   & 1.000 & 2.000 & 4.000  \\ \hline
6.000  & 600   & 1.200 & 2.400 & 4.800  \\ \hline
7.000  & 700   & 1.400 & 2.800 & 5.600  \\ \hline
8.000  & 800   & 1.600 & 3.200 & 6.400  \\ \hline
9.000  & 900   & 1.800 & 3.600 & 7.200  \\ \hline
10.000 & 1.000 & 2.000 & 4.000 & 8.000  \\ \hline
11.000 & 1.100 & 2.200 & 4.400 & 8.800  \\ \hline
12.000 & 1.200 & 2.400 & 4.800 & 9.600  \\ \hline
13.000 & 1.300 & 2.600 & 5.200 & 10.400 \\ \hline
14.000 & 1.400 & 2.800 & 5.600 & 11.200 \\ \hline
15.000 & 1.500 & 3.000 & 6.000 & 12.000 \\ \hline
\end{tabular}
\legend{Quantidade de Requisições Totais e Requisições Concorrentes}
\end{table}
Para cada valor de requisições totais, foram testados quatro cenários de acesso 
simultâneo: 10, 20, 40, e 80 porcento, totalizando 60 testes. Os dados obtidos 
nos testes foram tabelados em uma planilha eletrônica e analisados a partir de 
gráficos.

\section{Métricas utilizadas}
As métricas utilizadas para comparar o desempenho dos servidores HTTP são as 
mesmas métricas calculadas e entregues pelo ApacheBencch. São elas:
\begin{itemize}
	\item Tempo total do teste em segundos (s);
	\item Total de dados transferido em bytes (b);
	\item Total de texto em HTML transferido em bytes (b);
	\item Tempo médio por requisição em milissegundos (ms);
	\item Tempo médio por requisição entre as requisições concorrentes em 
	milissegundos (ms);
	\item Taxa de transferência em Quilo Bytes por segundo (Kb/s);
	\item Porcentagem das requisições servidas em um período de tempo, tempo em 
	milissegundos (X\%/ms).
\end{itemize}

\section{Organização dos dados}

\begin{itemize}
	\item[Faixa 1] Entre 1.000 e 5.000 requisições totais;
	\item[Faixa 2] Entre 6.000 e 10.000 requisições totais;
	\item[Faixa 3] Entre 11.000 e 15.000 requisições totais;
\end{itemize}


% ---
% Ambiente de Testes
% ---
\section{Ambiente de testes}

Para fazer os teste da forma mais neutra possível, foram criadas duas máquinas virtuais usando o software de virtualização VirtualBox na sua versão 3.4.10.\\
Todos os programas utilizados nos ambientes de teste foram instalados pelo gerenciador de pacotes nativo da distribuição Linux Debian \textit{aptitude} e disponíveis nos repositórios oficiais do Debian, exceto o SGBD PostgreSQL que foi instalado usando o \textit{aptitude} porém através do repositório oficial do PostgreSQL para o Linux Debian 7 pois nos repositórios do Debian não havia disponível a versão mais recente e utilizada pelo banco de dados do SIGA.

\subsection{Computador hospedeiro}
O computador onde foram instaladas as máquinas virtuais possui a seguinte 
configuração:
\begin{itemize}
	\item \textbf{Tipo}: Computador portátil (\textit{Notebook})
	\item \textbf{Processador}: Intel Core i5-3337U
	\item \textbf{Frequência do Processador}: 1,8 Giga Hertz
	\item \textbf{Tamanho da Memória Principal (RAM)}: 8 Giga Bytes;
	\item \textbf{Tamanho da Memória Secundária (HD)}: 1 Tera Bytes;
	\item \textbf{Quantidade de núcleos disponível para processamento}: 4 
	núcleos.
\end{itemize}

\subsection{Configurações comuns às duas máquinas virtuais}
 Ambas máquinas virtuais tinhas as configurações rigorosamente iguais. São elas:
\begin{itemize}
\item \textbf{Sistema Operacional}: Linux Debian 7 “Wheezy” 64 bits mais atualizado;
\item \textbf{Sistema Gerenciador de Banco de Dados}: PostgreSQL na versão 9.3.5
\item \textbf{Tamanho da Memória Principal (RAM)}: 2.048 Mega Bytes;
\item \textbf{Tamanho da Memória Secundária (HD)}: 30 Giga Bytes;
\item \textbf{Quantidade de núcleos disponível para processamento}: 1 núcleo.
\end{itemize}

\subsection{Máquina virtual Apache}
Na máquina virtual destinada aos testes com o servidor HTTP Apache, os programas utilizados foram:
\begin{itemize}
\item Apache HTTP Server na versão 2.2.22;
\item libapache2-mod-php5 na versão 5.4.4;
\item php5-common na versão 5.4.4.
\end{itemize}

\subsection{Máquina virtual Nginx}
Na maquina virtual destinada aos testes com o servidor HTTP Nginx, os programas utilizados foram:

\begin{itemize}
\item Servidor HTTP Nginx na versão 1.2.1;
\item php5-fpm na versão 5.4.4;
\item php5-common na versão 5.4.4.
\end{itemize}



% ---
% Análise dos dados
% ---
\chapter{Análise dos Dados}\label{cap:analise-dos-dados}

A análise dos dados será feita a partir dos dados coletados pelos testes realizados usando a ferramenta ApacheBench nas máquinas virtuais, cada uma utilizando um servidor HTTP diferente.\\
Os dados coletados nos testes foram tabelados e, para uma melhor análise, foram gerados gráficos dos mesmos.

\section{Algumas Considerações}
Com a grande quantidade de dados coletados, 75 para cada servidor HTTP totalizando 150 testes, cada teste gerando 8 parâmetros, gerando 1.200 dados, os dados serão agrupados de acordo com a quantidade total de requisições feitas no teste. \\
Os gráficos serão divididos em 3 faixas de valores para a quantidade total de requisições.
\begin{itemize}
	\item[Faixa 1] Entre 1.000 e 5.000 requisições totais;
	\item[Faixa 2] Entre 6.000 e 10.000 requisições totais;
	\item[Faixa 3] Entre 11.000 e 15.000 requisições totais;
\end{itemize}
Os dados serão analisados sempre comparando os resultados obtidos pelos dois servidores HTTP.\\

\section{Dados Analisados}
Os dados que serão analisados são:

\begin{itemize}
	\item Tempo total do teste em segundos (s);
	\item Total de dados transferido em Quilo bytes (Kbytes);
	\item Total de texto em HTML transferido em Quilo bytes (Kbytes);
	\item Número de requisições atendidas por segundo (X/s);
	\item Tempo médio por requisição em milissegundos (ms);
	\item Tempo médio por requisição entre as requisições concorrentes em milissegundos (ms);
	\item Taxa de transferência em Quilo Bytes por segundo (Kb/s);
	\item Porcentagem das requisições servidas em um período de tempo, tempo em milissegundos (X\%ms).
\end{itemize}

Dos dados analisados, os mais críticos são:

\begin{itemize}
	\item Total de texto em HTML transferido em bytes (b);
	\item Número de requisições atendidas por segundo (X/s);
	\item Tempo médio por requisição em milissegundos (ms);
	\item Tempo médio por requisição entre as requisições concorrentes em milissegundos (ms);
	\item Taxa de transferência em Quilo Bytes por segundo (Kb/s);
	\item Porcentagem das requisições servidas em um período de tempo, tempo em milissegundos (X\%ms).
\end{itemize}

Como foram coletados dados com 1 requisição concorrente, 10\%, 20\%, 40\% e 80\%, os gráficos serão criados utilizando as médias dos valores coletados, desprezando para o cálculo da média os testes feitos com uma requisição concorrente.

\section{Gráficos}
Todos os gráficos foram gerados usando a ferramenta de geração de gráfico do \citeonline{LOcalc}.
\subsection{Média do Tempo Total de Execução Dos Testes}
\subsubsection{Faixa 1}

blabla
\begin{figure}[htb]
	\centering
	\includegraphics[width=0.6\linewidth]{graficos/grafico1-f1} 
	\caption{Média do Tempo Total de Execução dos Testes - Faixa 1}
	\label{fig:grafico1-f1}
\end{figure}

blabla

\subsubsection{Faixa 2}
blabla

\begin{figure}[htb]
	\centering
	\includegraphics[width=0.6\linewidth]{graficos/grafico1-f2} 
	\caption{Média do Tempo Total de Execução dos Testes - Faixa 2}
	\label{fig:grafico1-f2}
\end{figure}
blabla

\subsubsection{Faixa 3}
blabla

\begin{figure}[htb]
	\centering
	\includegraphics[width=0.6\linewidth]{graficos/grafico1-f3} 
	\caption{Média do Tempo Total de Execução dos Testes - Faixa 3}
	\label{fig:grafico1-f3}
\end{figure}
blabla


\subsection{Média da Quantidade Total de Dados Transmitidos}
\subsubsection{Faixa 1}

blabla
\begin{figure}[htb]
	\centering
	\includegraphics[width=0.6\linewidth]{graficos/grafico2-f1} 
	\caption{Média do Total de Dados Transferidos - Faixa 1}
	\label{fig:grafico2-f1}
\end{figure}

blabla

\subsubsection{Faixa 2}
blabla

\begin{figure}[htb]
	\centering
	\includegraphics[width=0.6\linewidth]{graficos/grafico2-f2} 
	\caption{Média do Total de Dados Transferidos - Faixa 2}
	\label{fig:grafico2-f2}
\end{figure}
blabla

\subsubsection{Faixa 3}
blabla

\begin{figure}[htb]
	\centering
	\includegraphics[width=0.6\linewidth]{graficos/grafico2-f3} 
	\caption{Média do Total de Dados Transferidos - Faixa 3}
	\label{fig:grafico2-f3}
\end{figure}
blabla

\subsection{Média da Quantidade de Texto HTML Transmitido}
\subsubsection{Faixa 1}

blabla
\begin{figure}[htb]
	\centering
	\includegraphics[width=0.6\linewidth]{graficos/grafico3-f1} 
	\caption{Média do Total de Texto em HTML Transferido - Faixa 1}
	\label{fig:grafico3-f1}
\end{figure}

blabla

\subsubsection{Faixa 2}
blabla

\begin{figure}[htb]
	\centering
	\includegraphics[width=0.6\linewidth]{graficos/grafico3-f2} 
	\caption{Média do Total de Texto em HTML Transferido - Faixa 2}
	\label{fig:grafico3-f2}
\end{figure}
blabla

\subsubsection{Faixa 3}
blabla

\begin{figure}[htb]
	\centering
	\includegraphics[width=0.6\linewidth]{graficos/grafico3-f3} 
	\caption{Média do Total de Texto em HTML Transferido - Faixa 3}
	\label{fig:grafico3-f3}
\end{figure}
blabla


% ---
% Considerações Finais
% ---
\chapter{Considerações Finais}\label{cap:consideracoes-finais}

\section{Problemas conhecidos}

\section{Trabalhos futuros}
hhvm
nodejs
substituir do zero
problemas de atualização
jaspersoft

% ---
% Conclusão
% ---
\chapter{Conclusão}\label{cap:conclusao}

nodejs como servidor http
jaspereport - classe php para analisar jrxml
a morte do SIGA




% ----------------------------------------------------------
% ELEMENTOS PÓS-TEXTUAIS
% ----------------------------------------------------------
\postextual
% ----------------------------------------------------------

% ----------------------------------------------------------
% Referências bibliográficas
% ----------------------------------------------------------
\bibliography{bibliografia}

\end{document}

